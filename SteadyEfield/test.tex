
\documentclass{beamer}
\usetheme{Electromagnetism}
\usepackage{Electromagnetism}
\graphicspath{{pictures/}}
% -------------------------------------- Grid
%-------------------------------------------------------
\makeatletter
\def\grd@save@target#1{%
  \def\grd@target{#1}}
\def\grd@save@start#1{%
  \def\grd@start{#1}}
\tikzset{
  grid with coordinates/.style={
    to path={%
      \pgfextra{%
        \edef\grd@@target{(\tikztotarget)}%
        \tikz@scan@one@point\grd@save@target\grd@@target\relax
        \edef\grd@@start{(\tikztostart)}%
        \tikz@scan@one@point\grd@save@start\grd@@start\relax
        \draw[minor help lines] (\tikztostart) grid (\tikztotarget);
        \draw[major help lines] (\tikztostart) grid (\tikztotarget);
        \grd@start
        \pgfmathsetmacro{\grd@xa}{\the\pgf@x/1cm}
        \pgfmathsetmacro{\grd@ya}{\the\pgf@y/1cm}
        \grd@target
        \pgfmathsetmacro{\grd@xb}{\the\pgf@x/1cm}
        \pgfmathsetmacro{\grd@yb}{\the\pgf@y/1cm}
        \pgfmathsetmacro{\grd@xc}{\grd@xa + \pgfkeysvalueof{/tikz/grid with coordinates/major step}}
        \pgfmathsetmacro{\grd@yc}{\grd@ya + \pgfkeysvalueof{/tikz/grid with coordinates/major step}}
        \foreach \x in {\grd@xa,\grd@xc,...,\grd@xb}
        \node[anchor=north] at (\x,\grd@ya) {\pgfmathprintnumber{\x}};
        \foreach \y in {\grd@ya,\grd@yc,...,\grd@yb}
        \node[anchor=east] at (\grd@xa,\y) {\pgfmathprintnumber{\y}};
      }
    }
  },
  minor help lines/.style={
    help lines,
    step=\pgfkeysvalueof{/tikz/grid with coordinates/minor step}
  },
  major help lines/.style={
    help lines,
    line width= 0.5pt,
    step=\pgfkeysvalueof{/tikz/grid with coordinates/major step}
  },
  grid with coordinates/.cd,
  minor step/.initial=.2,
  major step/.initial=1,
  major line width/.initial=2pt,
}
\makeatother
\usepackage{cancel}
\begin{document}


% ============================== Слайд ## ===================================
\begin{frame}{Густини електричного заряду}{}
	\begin{columns}
		\begin{column}{0.7\linewidth}\centering
			\begin{block}{}\justifying\small
				Виділимо елементарний об'єм $dV$. Якщо в цьому об'ємі міститься заряд $dq$, то
				об'ємною густиною заряду називається величина:
				\begin{equation*}
					\rho = \frac{dq}{dV}
				\end{equation*}
			\end{block}
		\end{column}
		\begin{column}{0.3\linewidth}\centering
			\begin{tikzpicture}[>=latex,
					charge/.style = {ball color=red!50, font=\scriptsize, inner sep=1pt},
					surface/.style = {draw, opacity=0.5, smooth cycle, tension=.7},
					filled_surface/.style = {red, fill=red!50, smooth cycle, tension=.7},
					>=latex,
					scale=0.7,
					transform shape
				]

				\def\coordinates{
					(-1,0)
					(-0.5,1)
					(0.5,2)
					(2.5,1.5)
					(3,0.5)
					(2,-1.5)
					(0.5,-2)
					(-0.5,-1)
				}

				\begin{scope}[scale=0.7]
					\clip[yshift=2cm] plot[smooth cycle, tension=.7] coordinates \coordinates;
					\draw[yshift=2cm, draw=red, fill=red!40] plot[smooth cycle, tension=.7]
					coordinates \coordinates;
					\draw plot [only marks, mark=+, domain=-1:3, samples=100, mark
					options={color=red}]
					(\x,{rnd*4});
				\end{scope}
				\begin{scope}[shift={(55:2cm)}, line join=round]
					\pgfmathsetmacro{\cubex}{0.2}
					\pgfmathsetmacro{\cubey}{0.2}
					\pgfmathsetmacro{\cubez}{0.2}
					\draw[fill=red] (0,0,0) -- ++(-\cubex,0,0) -- ++(0,-\cubey,0) -- ++(\cubex,0,0) -- cycle;
					\draw[fill=red] (0,0,0) -- ++(0,0,-\cubez) -- ++(0,-\cubey,0) -- ++(0,0,\cubez) -- cycle;
					\draw[fill=red] (0,0,0) -- ++(-\cubex,0,0) -- ++(0,0,-\cubez) -- ++(\cubex,0,0) -- cycle;
				\end{scope}
				\node[font=\scriptsize, anchor=south west, inner sep=1pt, text
					width=2cm,
					align=center, text=blue] (div1) at (2.2,1) {Елемент об'єму \\ $dV$};
				\draw[<-, gray, thick] (55:2cm)
				to[out=45, in = 135] (div1.north west);
			\end{tikzpicture}
		\end{column}
	\end{columns}
	\begin{columns}
		\begin{column}{0.7\linewidth}
			\begin{block}{}\justifying\small
				Якщо заряд розподілений у тонкому шарі. Виділимо елемент шару
				площею $dS$, що несе заряд $dq$. Поверхневою густиною заряду називається
				величина:
				\begin{equation*}
					\sigma = \frac{dq}{dS}
				\end{equation*}
			\end{block}
		\end{column}
		\begin{column}{0.3\linewidth}\centering
			\begin{tikzpicture}[>=latex, every node/.style={font=\scriptsize}, scale=0.7,
					transform shape]
				\draw[fill=red!40, scale=0.7] (0,0) to[bend left] ++(2,1.9)  to[bend left]
				++(2,-2)
				to[bend right] ++(-2, -1)  to[bend right] cycle;
				\coordinate (A) at (1.3, 0.8);
				\draw[fill=red] (A) circle(0.2 and 0.1);
				\node[font=\scriptsize, anchor=south west, inner sep=1pt, text
					width=2.5cm,
					align=center, text=blue] (div1) at (2.5,1) {Елемент поверхні \\ $dS$};
				\draw[<-, gray, thick] (A)
				to[out=45, in = 135] (div1.north west);
			\end{tikzpicture}
		\end{column}
	\end{columns}
	\begin{columns}
		\begin{column}{0.7\linewidth}
			\begin{block}{}\justifying\small
				Лінійна густина як заряд, що
				припадає на одиницю довжини нитки (зразка, поперечні розміри якого малі
				порівняно з поздовжніми розмірами):
				\begin{equation*}
					\lambda = \frac{dq}{dl}
				\end{equation*}
			\end{block}
		\end{column}
		\begin{column}{0.3\linewidth}
			\begin{tikzpicture}[>=latex, every node/.style={font=\scriptsize}, scale=0.7, transform
			shape]
				\draw[red!40, line width=3pt] plot[domain=0.4:0.8*pi] ({\x}, {sin(\x r)});
				\draw[red, line width=3pt] plot[domain=1:1.4] ({\x}, {sin(\x r)});
				\node[font=\scriptsize, anchor=south west, inner sep=1pt, text width=2.5cm,
				align=center, text=blue] (div1) at (2.2,1) {Елемент довжини \\ $dl$};
                \draw[<-, gray, thick] (1.2, {sin(1.2 r)}) to[out=65, in = 135] (div1.north west);
			\end{tikzpicture}
		\end{column}
	\end{columns}
\end{frame}
% ===========================================================================


%\begin{tikzpicture}[>=latex, every node/.style={font=\scriptsize}]
%    \draw[ball color=blue!5] (0,0) to[bend left] ++(2,1.9)  to[bend left] ++(2,-2)  to[bend right]
%    ++(-2, -1)  to[bend
%    right] cycle;
%    \draw[->, red] (0.2, 0.2) -- ++(80:1.5);
%    \draw[->, red] (0.8, 1) -- ++(80:1.5);
%    \draw[->, red] (1.6, 1.6) -- ++(80:1.5);
%
%    \draw[->, red] (1.2, 0) -- ++(75:1.5);
%    \draw[->, red] (2.2, 0.8) coordinate (A) -- ++(70:1.5) coordinate (B) node[above] {$\vect{v}$};
%    \draw[->, red] (3, 1.2) -- ++(70:1.5);
%
%    \draw[->, red] (2, -0.4) -- ++(45:1.5);
%    \draw[->, red] (2.8, 0) -- ++(45:1.5);
%    \draw[->, red] (3.8, 0.2) -- ++(35:1.5);
%
%    \draw[fill=gray!50, opacity=0.5] (A) circle(0.5 and 0.15);
%    \draw[dashed] (A) ++(70:1.5) circle(0.5 and 0.15);
%    \draw[->, green!50!black, thick] (A) -- ++(90:0.75) node[above, text=black] {$\vect{n}$};
%    \draw[densely dashed] (A) ++(180:0.5) -- ++(70:1.5);
%    \draw[densely dashed] (A) ++(0:0.5) -- ++(70:1.5);
%\end{tikzpicture}


% ===========================================================================




% ============================== Слайд ## ===================================
%\begin{frame}{Приклади розрахунку потоків}{}
%
%	\begin{columns}
%		\begin{column}{0.5\linewidth}\centering
%			\begin{tikzpicture}[>=latex]
%
%
%				\shadedraw [ball color=gray!5,shading=ball,opacity=1,  ultra thin] (0,0) ellipse
%(0.5cm and
%				1cm);
%
%				\begin{scope}
%					\node[circle, ball color=red, text=white, inner sep=0.25pt, font=\scriptsize]
%(Q) at
%					(0.1,0)
%					{$+$};
%					\clip  (0.1,0) ellipse (0.5cm and 1cm);
%					\foreach \a in {0,20,...,340}{
%							\draw[red] (Q) -- (\a:1.1);
%						}
%				\end{scope}
%				\shadedraw [ball color=gray!5,shading=ball,opacity=1, ultra thin] (0,-1) arc
%(-90:90:1cm and
%				10mm)
%				arc
%				(90:-90:0.5cm and 1cm);
%
%				%    \foreach \a in {-80,-70,...,80}{
%				%        \draw[red, ->] (0,0) ++(\a:1) -- ++(\a:{0.3)});
%				%    }
%				\pgfmathsetseed{9}
%				\foreach \angle in {70, 60, ..., -70} {
%						\pgfmathsetmacro\start{0.7 + abs(rand*(1-0.7))}  % Генерация случайного
%%числа от
%						%0.6 до 1
%						\pgfmathsetmacro\end{1.9 -\start}            % Вычисление \end так, чтобы
%%\start +
%						%\end
%						%= 1.6
%						\draw[red, midarrow] (0,0) ++(\angle:\start) -- ++(\angle:\end);
%					}
%
%
%			\end{tikzpicture}
%		\end{column}
%		\begin{column}{0.5\linewidth}\centering
%
%		\end{column}
%	\end{columns}
%
%\end{frame}
% ===========================================================================





\end{document}