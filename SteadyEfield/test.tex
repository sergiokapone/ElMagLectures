
\documentclass{beamer}
\usetheme{Electromagnetism}
\usepackage{Electromagnetism}
\graphicspath{{pictures/}}
% -------------------------------------- Grid
%-------------------------------------------------------
\makeatletter
\def\grd@save@target#1{%
  \def\grd@target{#1}}
\def\grd@save@start#1{%
  \def\grd@start{#1}}
\tikzset{
  grid with coordinates/.style={
    to path={%
      \pgfextra{%
        \edef\grd@@target{(\tikztotarget)}%
        \tikz@scan@one@point\grd@save@target\grd@@target\relax
        \edef\grd@@start{(\tikztostart)}%
        \tikz@scan@one@point\grd@save@start\grd@@start\relax
        \draw[minor help lines] (\tikztostart) grid (\tikztotarget);
        \draw[major help lines] (\tikztostart) grid (\tikztotarget);
        \grd@start
        \pgfmathsetmacro{\grd@xa}{\the\pgf@x/1cm}
        \pgfmathsetmacro{\grd@ya}{\the\pgf@y/1cm}
        \grd@target
        \pgfmathsetmacro{\grd@xb}{\the\pgf@x/1cm}
        \pgfmathsetmacro{\grd@yb}{\the\pgf@y/1cm}
        \pgfmathsetmacro{\grd@xc}{\grd@xa + \pgfkeysvalueof{/tikz/grid with coordinates/major step}}
        \pgfmathsetmacro{\grd@yc}{\grd@ya + \pgfkeysvalueof{/tikz/grid with coordinates/major step}}
        \foreach \x in {\grd@xa,\grd@xc,...,\grd@xb}
        \node[anchor=north] at (\x,\grd@ya) {\pgfmathprintnumber{\x}};
        \foreach \y in {\grd@ya,\grd@yc,...,\grd@yb}
        \node[anchor=east] at (\grd@xa,\y) {\pgfmathprintnumber{\y}};
      }
    }
  },
  minor help lines/.style={
    help lines,
    step=\pgfkeysvalueof{/tikz/grid with coordinates/minor step}
  },
  major help lines/.style={
    help lines,
    line width= 0.5pt,
    step=\pgfkeysvalueof{/tikz/grid with coordinates/major step}
  },
  grid with coordinates/.cd,
  minor step/.initial=.2,
  major step/.initial=1,
  major line width/.initial=2pt,
}
\makeatother
\usepackage{cancel}
\begin{document}

% ============================== Слайд ## ===================================
\begin{frame}{Лекційні задачі \No 3}

\begin{overprint}
\onslide<1>
\begin{problem}[Задача 1]
Однорідно заряджена куля радіусом $R$ має сумарний заряд $Q$.

\begin{enumerate}
  \item Знайдіть вираз для електростатичного потенціалу $\varphi(r)$ як функції відстані $r$ від центра кулі для двох випадків:
  \begin{itemize}
    \item всередині кулі ($0 \le r \le R$);
    \item зовні кулі ($r \ge R$),
  \end{itemize}
  при умові $\varphi(\infty)=0$.

  \item Побудуйте графік залежності $\varphi(r)$ від $r$ на відрізку $0 \le r \le 2R$.
  На графіку позначте значення потенціалу в центрі ($r=0$) та на поверхні кулі ($r=R$).
\end{enumerate}
\end{problem}

\onslide<2>
\begin{problem}[Задача 2]
Розглянемо електричний диполь з дипольним моментом
\[
\vec{p} = p\ \vec{e}_z,
\]
розташований у неоднорідному електричному полі
\[
\vec{E}(x,y,z) = C \, z \, \vec{e}_z,
\]
де $C= \text{const}$.

\begin{enumerate}
  \item Обчисліть силу $\vec{F}$, що діє на диполь, використовуючи вираз
  \[
  \vec{F} = (\vec{p}\cdot\nabla)\vec{E}.
  \]
  \item Побудуйте якісну схему силових ліній та вкажіть, у який бік рухатиметься диполь.
\end{enumerate}
\end{problem}
\end{overprint}

\end{frame}
\end{document}