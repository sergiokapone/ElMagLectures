
% !TeX program = lualatex
% !TeX encoding = utf8
% !TeX spellcheck = uk_UA

\documentclass[14pt]{extarticle}

\usepackage{fontspec}
\setsansfont{CMU Sans Serif}%{Arial}
\setmainfont{CMU Serif}%{Times New Roman}
\setmonofont{CMU Typewriter Text}%{Consolas}
\defaultfontfeatures{Ligatures={TeX}}
\usepackage[math-style=TeX]{unicode-math}
\usepackage[english, russian, ukrainian]{babel}


\usepackage[%
	a4paper,%
	footskip=1cm,%
	headsep=0.3cm,%
	top=2cm, %поле сверху
	bottom=2cm, %поле снизу
	left=2cm, %поле ліворуч
	right=2cm, %поле праворуч
    ]{geometry}

\renewcommand{\baselinestretch}{1}


\setlength{\parskip}{0.5ex}%
\setlength{\parindent}{2.5em}%

\usepackage{amsmath}
\usepackage{graphicx}
\usepackage{floatflt}

\usepackage[%colorlinks=true,
	%urlcolor = blue, %Colour for external hyperlinks
	%linkcolor  = malina, %Colour of internal links
	%citecolor  = green, %Colour of citations
	bookmarks = true,
	bookmarksnumbered=true,
	unicode,
	linktoc = all,
	hypertexnames=false,
	pdftoolbar=false,
	pdfpagelayout=TwoPageRight,
	pdfauthor={Ponomarenko S.M. aka sergiokapone},
	pdfdisplaydoctitle=true,
	pdfencoding=auto
	]%
	{hyperref}
		\makeatletter
	\AtBeginDocument{
	\hypersetup{
		pdfinfo={
		Title={\@title},
		}
	}
	}
	\makeatother

\title{}
\author{}



\begin{document}


\section{Експериментальні перевірки закону Кулона}

Формула (1.1.1) перевірялася Кулоном у дослідах, де прямо визначалася сила взаємодії зарядів за допомогою крутильних терезів.

Якщо припустити, що закон взаємодії точкових зарядів відрізняється від (1.1.1), наприклад:
\[
F = k \frac{q_1 q_2}{R^{2 + \varepsilon}},
\]
де $\varepsilon$ — мала величина, а $|\varepsilon| \ll 1$. При збереженні принципу суперпозиції це призводить до якісних змін у поведінці зарядів у
провідниках. У випадку кулонівської взаємодії (1.1.1) у стаціонарних умовах заряди зосереджуються на поверхні провідника, а всередині зарядів немає.
Однак, якщо виконується формула (7), можна показати, що у провіднику виникає ненульова густина заряду у стаціонарних умовах.

Цю обставину було використано для отримання обмеження на $\varepsilon$. На попередньо заряджену провідну кулю накладали дві провідні напівсфери, які
щільно прилягали одна до одної, утворюючи майже суцільну поверхню. У цій системі, у разі кулонівської взаємодії ($\varepsilon = 0$), весь заряд
переходить на зовнішні напівсфери. Якщо ж $\varepsilon \neq 0$, всередині кулі залишається заряд (тим менший, чим менше $\varepsilon$), який можна
виміряти за допомогою чутливого електрометра після зняття напівсфер. Цей експеримент неодноразово повторювався із поступовим підвищенням точності. У
1983 році було отримано обмеження:
\[
|\varepsilon| < 10^{-16} \text{–} 10^{-17}.
\]

Формула (7) відповідає потенціалу статичного точкового заряду:
\[
\varphi(r) = \frac{q}{r^{1 + \varepsilon}},
\]
де $r$ — відстань до заряду. Можна також розглянути іншу залежність потенціалу:
\[
\varphi(r) = k \frac{q}{r} e^{-\mu r},
\]
яка виникає у теорії поля з ненульовою масою покою фотона. Тут $\mu = m_\gamma c / \hbar$, де $\hbar$ — стала Планка, а $m_\gamma$ — маса фотона.
Величина $\mu$ обмежена на підставі досліджень електромагнітних хвиль великої довжини та магнітних полів у Сонячній системі:
\[
\mu < 1 / 10^{10} \, \text{см}^{-1}.
\]
Це дає обмеження для маси фотона:
\[
m_\gamma < 3 \cdot 10^{-26} \, m_e,
\]
де $m_e$ — маса електрона.

\section{Перевірка закону квантування заряду}

Як вже зазначалося, заряд протона за абсолютною величиною дорівнює заряду електрона, що узгоджується з фактом електронейтральності атомів. Завдяки цьому
ми зазвичай маємо справу з нейтральною речовиною. Однак нейтральність речовини не доводить нейтральності кожного окремого атома. Наприклад, якщо атоми
мають невеликий заряд, середній заряд макроскопічних тіл може компенсуватися іонами протилежного знаку, що завжди є на поверхні тіл у повітрі тощо.

Щоб перевірити рівність зарядів електрона та протона, вимірювався повний заряд при витіканні аргону зі спеціальної камери у вакуумі. Вільні іони
усувалися за допомогою пастки з поперечним електричним полем: заряд іонів значно більший за гіпотетичний заряд «майже нейтральних» атомів аргону, тому
саме іони реагували на електричне поле пастки. Результати 1959 року дали таку оцінку різниці зарядів протона та електрона:
\[
\left| \frac{\Delta e}{e} \right| < 2 \cdot 10^{-21}.
\]
В інших експериментах досліджувалося можливе відхилення пучка нейтральних атомів в електричному полі (див. И.Н. Мешков, Б.В. Чириков, ч. 1, §13).



\end{document}


