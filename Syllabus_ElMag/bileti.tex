%%============================ Compiler Directives =======================%%
%%                                                                        %%
% !TeX program = pdflatex
% !TeX encoding = utf8
% !TeX spellcheck = uk_UA
%%                                                                        %%
%%============================== Клас документа ==========================%%
%%                                                                        %%
\documentclass[14pt,oneside]{extbook}
%%                                                                        %%
%%============================= Мови та кодування ========================%%
%%                                                                        %%
\usepackage[utf8]{inputenc}
\usepackage[T2A,T1]{fontenc}
\usepackage[ukrainian]{babel}
%%                                                                        %%
%%=========================== Киририличні корекції =======================%%
\usepackage{indentfirst}
\usepackage{misccorr}
\usepackage{cmap}
%%                                                                        %%
%%============================= Геометрія сторінки =======================%%
%%                                                                        %%
\usepackage[a4paper,
			top=1cm, %поле сверху
			bottom=1cm, %поле снизу
			left=1cm, %поле справа
			right=1cm]{geometry}%поле слева

%%============================== Додаткові пакети ========================%%
%%                                                                        %%
\usepackage[version=4]{mhchem}
\usepackage{amsmath}
\usepackage{pifont}
%\usepackage{pscyr}
\usepackage[many]{tcolorbox}
\usepackage{amsmath}
\usepackage{readarray}
%%                                                                        %%
%%========================================================================%%

\renewcommand\labelenumi{\theenumi.}
\newcounter{bilets}
\setcounter{bilets}{1}
\def\textbfc#1{\color{black}{\textbf{#1}}}

\newcommand{\signature}[5]{%
	% #1 = total width
	% #2 = padding on the right
	% #3 = text on the left
	% #4 = text above the rule
	% #5 = text below the rule
	\par\noindent
	\makebox[#1][s]{#3\ \lowrule{\fill}\makebox[0pt]{#4}\lowrule{#2}}\\[-10pt]
	\makebox[#1][s]{\hphantom{#3}\ \hfill\makebox[0pt]{\scriptsize#5}\hspace{#2}}%
}

\newcommand*\signline[3]{\tikz[baseline]{
    \node[inner sep=#3] (char) {#1};
    \draw[](char.base west)--(char.base east);
    \node[below] (char) {\footnotesize #2};
    }%
    }

\newcommand{\lowrule}[1]{%
	\leaders\hrule height \dimexpr-\dp\strutbox+0.4pt\relax depth \dp\strutbox\hskip#1\relax
}

\def\Cut{%
\vskip1ex
\noindent\dotfill%
\vfill
}

\def\signatureline#1#2#3{%
	\raisebox{0.25\baselineskip}{$\overset{\text{\normalsize #1}}{\underset{\text{\scriptsize #2}}{\underline{\hspace{#3}}}}$}%
}
\newcommand{\signat}{ \raisebox{-\baselineskip}{\shortstack{\underline{\hspace{5cm}}\\ \scriptsize (підпис)}}}
\def\zatv#1{\hbox{\small \texttt{Затверджено на засіданні кафедри\, #1}}}
\def\protocol#1#2#3#4{\hbox{\small \texttt{Затверджено на засіданні кафедри, протокол №#1 від <<#2>> #3 #4 р.}}}
\def\zavkaf{\footnotesize \textbf{Завідувач кафедри:} \signatureline{}{(підпис)}{2cm} \signatureline{\footnotesize Монастирський Г.~Є.}{(прізвище,
ініціали)}{3cm} \textbf{Екзаменатор:} \signatureline{}{(підпис)}{2cm} \signatureline{\footnotesize Пономаренко С.~М.}{(прізвище, ініціали)}{3cm} }


%------------------------------------------------------------------------------------------


	\newtcolorbox{bilety}
	{   enhanced,
		%watermark tikz={\draw[color=black!10!white,line width=2mm] circle (1cm)
		%	node{\fontseries{b}\fontsize{20mm}{20mm}\selectfont ?};},
		space to upper,
		height=13.1cm,
		%borderline={0.3mm}{0mm}{black!75, dashed},
		segmentation style={black, solid, opacity=0, line width=0pt},
		colback =  white,
		colframe = black!15!white,
		sharpish corners,
	}

%%================================================== Верстка білета (begin) ======================================================%%
\newenvironment{bilet}{%
	\noindent
	\begin{bilety}
		\signatureline{\footnotesize \bfseries КПІ ім. Ігоря Сікорського}{(назва вищого навчального закладу)}{10cm} \hfill \signatureline{\footnotesize \bfseries Форма \No У -- 5.09}{}{5cm}  		\\
		\signature{\linewidth}{.4\linewidth}{Спеціальність: }{105 Прикладна фізика та наноматеріали}{}
		\signature{\linewidth}{.4\linewidth}{Семестр: }{3}{}
		\signature{\linewidth}{.4\linewidth}{Навчальний предмет: }{Електрика та магнетизм}{}
	\begin{center}
		\large\textbf{Екзаменаційний білет № \thebilets}
	\end{center}
		\begin{enumerate}}
		{\end{enumerate}
		\vbox{\vfill}
		%\hrule
		\tcblower
%		\zatv{Прикладної фізики}
		\protocol{13}{10}{грудня}{2025}
		\smallskip
		\hbox{\zavkaf}
	\end{bilety}%
	\ifodd\thebilets\Cut\fi%
	\addtocounter{bilets}{1}%
}%
%% =========================================== Верстка білета (end) ==================================================== %%

\AtBeginDocument{\pagestyle{empty}}

\newcounter{rand}
\def\RandData#1{
\readrecordarray{#1}\data
\setcounter{rand}{\pdfuniformdeviate\dataROWS{}}
\ifnum\therand=0\stepcounter{rand}\fi
\data[\arabic{rand}]
}


\newcommand\Repeat[2]{
 \newcounter{i}
  \loop \ifnum\value{i} < #1
    #2%
    \stepcounter{i}%
  \repeat
}

\begin{document}

%\Repeat{20}{
%\begin{bilet}%\small
%\item \RandData{ElMagQ.dat}
%\item \RandData{ElMagQ.dat}
%\end{bilet}
%}


\begin{bilet}
\item Закон Кулона: формулювання, область застосування, одиниці.
\item Магнітне поле стаціонарної системи струмів на великих відстанях. Магнітні полюси.
\item Тіло заряджене рівномірно лінійно з густиною $\lambda=3\cdot10^{-6}$~Фр/см. Знайдіть напруженість на відстані $5$ см.
\end{bilet}

\begin{bilet}
\item Електростатичне поле: визначення, напруженість, суперпозиція.
\item Потенціальна енергія диполя в магнітному полі. Сила на диполь.
\item Нескінченний дріт несе струм $I=3$ Гаусс$\cdot$см. Знайдіть $H$ на відстані $2$ см. Розв'яжіть задачу двома способами: через закон Біо-Мавара-Лапласа, та використовуючи теорему про циркуляцію вектора $\vec{B}$.
\end{bilet}

\begin{bilet}
\item Теорема Гауса: формулювання та застосування до нескінченної нитки, площини, сфери.
\item Закон електромагнітної індукції Фарадея. Вихрове електричне поле.
\item Амплітуда плоскої гармонічної електромагнітної хвилі у вакуумі має напруженість $E_0=5\cdot10^2$ статВ/см. Знайдіть амплітуду індукції магнітного поля.
\end{bilet}

\begin{bilet}
\item Потенціал електричного поля. Зв’язок $\vec{E}$ і $\varphi$.
\item Комплексний опір. Закони Кірхгофа для змінного струму.
\item Плоский конденсатор має площу пластин $50$ см$^2$ і відстань між ними $0.1$ см. Визначіть його ємність.
\end{bilet}

\begin{bilet}
\item Градієнт, дивергенція, ротор. Теореми Стокса й Остроградського–Гаусса.
\item Магнітне поле в речовині. Намагніченість, гіпотеза Ампера.
\item Диполь з моментом $p=2\cdot10^{-6}$ стКл$\cdot$см знаходиться у однорідному полі $E=300$ статВ/см під кутом $30^\circ$ до напряму силових лінії. Знайдіть момент сил, що діє на цей диполь та енергію диполя.
\end{bilet}

\begin{bilet}
\item Дипольний момент. Момент сил у однорідному полі.
\item Скін-ефект. Глибина проникнення змінного поля.
\item Для міді з питомим опором $\rho = 1\cdot10^{-6}\ \text{с}$ та частотою струму $\omega = 10^{6}\ \text{c}^{-1}$ обчисліть глибину проникнення поля.
\end{bilet}

\begin{bilet}
\item Провідники в електростатичному полі. Електростатична індукція.
\item Магнітний момент. Гіромагнітне відношення.
\item Хвиля має амплітуду напруженості електричного поля $E_0=500$ статВ/см. Знайдіть середню густину енергії хвилі.
\end{bilet}

\begin{bilet}
\item Рівняння Пуассона та Лапласа. Принцип єдиності розв’язку.
\item Закон збереження енергії електромагнітного поля. Вектор Пойнтінга.
\item Соленоїд з $n=200$ витків/см несе струм $I=0.01$ стА. Знайдіть напруженість магнітного поля в середині соленоїда.
\end{bilet}

\begin{bilet}
\item Метод електричних зображень: суть і застосування.
\item Рух зарядженої частинки в однорідному магнітному полі. Робота магнітного поля.
\item Електрон рухається перпендикулярно до однорідного магнітного поля $B = 50$ Гс. Знайдіть циклотронну частоту його обертання.
\end{bilet}

\begin{bilet}
\item Взаємна ємність провідників. Конденсатори та енергія електричного поля.
\item Само- й взаємоіндукція. Перехідні процеси в колах з індуктивністю.
\item Точковий заряд коливається з амплітудою $a=1$ см і частотою $\omega = 10^{8}\ \text{c}^{-1}$. Оцініть потужність випромінювання цього заряду.
\end{bilet}

\begin{bilet}
\item Поляризація діелектриків. Вільні й зв’язані заряди.
\item Теорема про циркуляцію вектора $\vec{B}$ у середовищі.
\item Яку мінімальну швидкість повинен мати електрон, який знаходиться на відстані $4R$ від металевої сфери радіуса $R$, щоб він міг досягти її
поверхні. Сфера заряджена до потенціалу $400$~В. Маса електрона $9,1093826(16)\cdot 10^{-31}$~кг, заряд $-1,6021892(46)\cdot 10^{-19}$~Кл.
\end{bilet}

\begin{bilet}
\item Вектор електричної індукції $\vec{D}$ і закон Гауса в діелектриках.
\item Магнітна проникність. Діа-, пара-, феромагнетики.
\item Знайти магнiтне поле, що створюється тонким пiвкiльцем радiусом $R$ в його центрi.
\end{bilet}

\begin{bilet}
\item Діелектрична проникність та її температурна залежність.
\item Дивергенції полів $\vec{B}$ і $\vec{H}$, граничні умови.
\item  У плоский повiтряний конденсатор паралельно до обкладок вставляють пластину
дiелектрика з проникнiстю $\varepsilon$ i товщиною в половину зазору мiж обкладками. Як i в скiльки разiв
змiниться ємнiсть конденсатора?
\end{bilet}

\begin{bilet}
\item Енергія електростатичного поля та пондеромоторні сили.
\item Вектор-потенціал $\vec{A}$. Калібрувальна інваріантність.
\item  Згiдно теорiї Бора, електрон в основному станi атома водню обертається навколо
ядра по коловiй орбiтi на вiдстанi, що дорiвнює борiвському радiусу $a$. Знайти магнiтний момент
електрона, що пов’язаний з таким орбiтальним рухом.
\end{bilet}

\begin{bilet}
\item Закон Ома в інтегральній і диференціальній формах. Закони Кірхгофа.
\item Потік вектора $\vec{B}$. Теорема Гаусса для магнітного поля.
\item У колі, індуктивність якого $L=5\cdot10^{-6}$ Гн струм зростає зі швидкістю $10^7$~стА/с. Знайдіть ЕРС самоіндукції.
\end{bilet}

\begin{bilet}
\item Закон Ома для повного кола. Електро рушійна сила.
\item Магнітне поле: основні властивості та фізичний зміст вектора  $\vec{B}$.
\item Через резистор опором $R = 200$~Ом протікає змінний струм амплітудою $I_{0} = 0.03\ \text{A}$. Визначте потужність, що виділяється на резисторі.
\end{bilet}

\begin{bilet}
\item Закон Джоуля–Ленца в інтегральній та диференціальній формі. Нагрівання провідників.
\item Інваріанти електромагнітного поля. Відносність $\vec{E}$ і $\vec{B}$.
\item Електромагнітна хвиля поширюється у середовищі з $\varepsilon=4$ та $\mu=2$. Знайдіть її фазову швидкість та абсолютний показник заломлення.
\end{bilet}

\begin{bilet}
\item Рівняння Максвелла. Струм зміщення. Матеріальні рівняння.
\item Монохроматичні плоскі електромагнітні хвилі. Швидкість і показник заломлення.
\item Два паралельні дроти несуть струми $I$ та $2I$ і розташовані на відстані $3$ см один від одного. Точка спостереження лежить на відстані $4$ см від першого дроту і $5$ см від другого. Визначте результуючу індукцію магнітного поля в цій точці.
\end{bilet}

\begin{bilet}
\item Імпульс та момент імпульсу електромагнітного поля. Тиск хвилі.
\item Резонанс у колах змінного струму.
\item Однорідно заряджений стрижень довжиною $l = 10\ \text{см}$ і повним зарядом $Q = 5\cdot10^{-6}\ \text{Фр}$ рівномірно обертається навколо своєї осі з частотою $f = 50\ \text{Гц}$. Визначте його магнітний момент.
\end{bilet}

\begin{bilet}
\item Абсолютний показник заломлення. Фазова й групова швидкість хвиль.
\item Трифазний струм. Генератори, двигуни, трансформатори.
\item Електричний диполь коливається з однаковою амплітудою прискорення при двох різних частотах: $\omega_{1} = 10^{7}\ \text{c}^{-1}$ та $\omega_{2} = 2\cdot10^{7}\ \text{c}^{-1}$. У скільки разів відрізняється потужність його випромінювання?
\end{bilet}



\end{document}
