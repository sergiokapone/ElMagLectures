
\documentclass{beamer}
\usetheme{Electromagnetism}
\usepackage{Electromagnetism}
\graphicspath{{pictures/}}
% -------------------------------------- Grid
%-------------------------------------------------------
\makeatletter
\def\grd@save@target#1{%
  \def\grd@target{#1}}
\def\grd@save@start#1{%
  \def\grd@start{#1}}
\tikzset{
  grid with coordinates/.style={
    to path={%
      \pgfextra{%
        \edef\grd@@target{(\tikztotarget)}%
        \tikz@scan@one@point\grd@save@target\grd@@target\relax
        \edef\grd@@start{(\tikztostart)}%
        \tikz@scan@one@point\grd@save@start\grd@@start\relax
        \draw[minor help lines, gray!50] (\tikztostart) grid (\tikztotarget);
        \draw[major help lines, gray!50] (\tikztostart) grid (\tikztotarget);
        \grd@start
        \pgfmathsetmacro{\grd@xa}{\the\pgf@x/1cm}
        \pgfmathsetmacro{\grd@ya}{\the\pgf@y/1cm}
        \grd@target
        \pgfmathsetmacro{\grd@xb}{\the\pgf@x/1cm}
        \pgfmathsetmacro{\grd@yb}{\the\pgf@y/1cm}
        \pgfmathsetmacro{\grd@xc}{\grd@xa + \pgfkeysvalueof{/tikz/grid with coordinates/major step}}
        \pgfmathsetmacro{\grd@yc}{\grd@ya + \pgfkeysvalueof{/tikz/grid with coordinates/major step}}
        \foreach \x in {\grd@xa,\grd@xc,...,\grd@xb}
        \node[anchor=north] at (\x,\grd@ya) {\pgfmathprintnumber{\x}};
        \foreach \y in {\grd@ya,\grd@yc,...,\grd@yb}
        \node[anchor=east] at (\grd@xa,\y) {\pgfmathprintnumber{\y}};
      }
    }
  },
  minor help lines/.style={
    help lines,
    step=\pgfkeysvalueof{/tikz/grid with coordinates/minor step}
  },
  major help lines/.style={
    help lines,
    line width= 0.5pt,
    step=\pgfkeysvalueof{/tikz/grid with coordinates/major step}
  },
  grid with coordinates/.cd,
  minor step/.initial=.2,
  major step/.initial=1,
  major line width/.initial=2pt,
}
\makeatother

\begin{document}



% ============================== Слайд ## ===================================
\begin{frame}{}{}
\begin{tikzpicture}[>=latex]
    % Основные параметры
    \pgfmathsetmacro\h{3}          % Высота
    \pgfmathsetmacro\R{1}          % Радиус
    \coordinate (A) at (0, {-\h/2});
    \coordinate (B) at (0, {+\h/2});

    % Вертикальные линии
    \draw[gray] ([xshift=\R cm]A) -- ([xshift=\R cm]B)
                ([xshift=-\R cm]A) -- ([xshift=-\R cm]B);
    \draw[line width={\R*2cm}, gray!50] (A) -- (B);

    % Верхняя и нижняя части
    \draw[gray, fill=gray!50] (B) circle ({\R} and {0.2*\R});
    \draw[gray] ([xshift=\R cm]A) arc(0:-180:{\R} and {0.2*\R});
    \draw[gray, densely dashed] ([xshift=-\R cm]A) arc(180:0:{\R} and {0.2*\R});
    \fill[gray!50] (A) circle ({\R} and {0.2*\R});

    % Стрелка для тока
    \draw[->] ([yshift={\h*0.3cm}]A) -- ([yshift={-\h*0.3cm}]B) node[above] {$I$};

    % Параметры для витков
    \def\fLarge{0.8}  % Масштаб большого витка
    \def\fSmall{0.4}  % Масштаб маленького витка

    % Витки на разных уровнях
    \foreach \y in {-1.5, -1, ..., +1.5} {
        \pgfmathtruncatemacro{\c}{round(\y)}
        \ifnum\c=0 \edef\t{1} \else \edef\t{0.1} \fi % Прозрачность
        \begin{scope}[opacity=\t]

            % Большой виток
            \draw[arrowpos={0.4}{0}{3pt}, blue] (0,{0.8*\y})
                [partial ellipse=91:{360+89}:{\fLarge*\R} and {\fLarge*0.2*\R}];
            \draw[arrowpos={0.5}{2pt}{3pt}, green!50!black] (0,{0.8*\y})
                ++(180:{\fLarge*\R}) circle(0.2);
            \draw[arrowpos={0}{2pt}{3pt}, green!50!black] (0,{0.8*\y})
                ++(0:{\fLarge*\R}) circle(0.2);

            % Маленький виток
            \draw[arrowpos={0.4}{0}{2pt}, blue] (0,{0.8*\y})
                [partial ellipse=91:{360+89}:{\fSmall*\R} and {\fSmall*0.2*\R}];
            \draw[arrowpos={0.5}{1.2pt}{2pt}, green!50!black] (0,{0.8*\y})
                ++(180:{\fSmall*\R}) circle(0.1);
            \draw[arrowpos={0.5}{1.2pt}{2pt}, green!50!black, rotate=180] (0,{0.8*\y})
                ++(180:{\fSmall*\R}) circle(0.1);
        \end{scope}
    }
\end{tikzpicture}
\end{frame}
% ===========================================================================



\end{document}
