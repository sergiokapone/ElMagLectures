
\documentclass{beamer}
\usetheme{Electromagnetism}
\usepackage{Electromagnetism}
\graphicspath{{pictures/}}
% -------------------------------------- Grid
%-------------------------------------------------------
\makeatletter
\def\grd@save@target#1{%
  \def\grd@target{#1}}
\def\grd@save@start#1{%
  \def\grd@start{#1}}
\tikzset{
  grid with coordinates/.style={
    to path={%
      \pgfextra{%
        \edef\grd@@target{(\tikztotarget)}%
        \tikz@scan@one@point\grd@save@target\grd@@target\relax
        \edef\grd@@start{(\tikztostart)}%
        \tikz@scan@one@point\grd@save@start\grd@@start\relax
        \draw[minor help lines, gray!50] (\tikztostart) grid (\tikztotarget);
        \draw[major help lines, gray!50] (\tikztostart) grid (\tikztotarget);
        \grd@start
        \pgfmathsetmacro{\grd@xa}{\the\pgf@x/1cm}
        \pgfmathsetmacro{\grd@ya}{\the\pgf@y/1cm}
        \grd@target
        \pgfmathsetmacro{\grd@xb}{\the\pgf@x/1cm}
        \pgfmathsetmacro{\grd@yb}{\the\pgf@y/1cm}
        \pgfmathsetmacro{\grd@xc}{\grd@xa + \pgfkeysvalueof{/tikz/grid with coordinates/major step}}
        \pgfmathsetmacro{\grd@yc}{\grd@ya + \pgfkeysvalueof{/tikz/grid with coordinates/major step}}
        \foreach \x in {\grd@xa,\grd@xc,...,\grd@xb}
        \node[anchor=north] at (\x,\grd@ya) {\pgfmathprintnumber{\x}};
        \foreach \y in {\grd@ya,\grd@yc,...,\grd@yb}
        \node[anchor=east] at (\grd@xa,\y) {\pgfmathprintnumber{\y}};
      }
    }
  },
  minor help lines/.style={
    help lines,
    step=\pgfkeysvalueof{/tikz/grid with coordinates/minor step}
  },
  major help lines/.style={
    help lines,
    line width= 0.5pt,
    step=\pgfkeysvalueof{/tikz/grid with coordinates/major step}
  },
  grid with coordinates/.cd,
  minor step/.initial=.2,
  major step/.initial=1,
  major line width/.initial=2pt,
}
\makeatother

\begin{document}



% ============================== Слайд ## ===================================
\begin{frame}{Зв'язок намагніченості з молекулярними струмами}{}


	\begin{block}{}\justifying\small
		Виділимо в речовині досить малий циліндр, так що поле в ньому можна вважати практично однорідним. У його об'ємі молекулярні струми компенсують один
		одного. Циліндр (ліворуч) і вигляд його торця (праворуч). Кільцеві струми, що циркулюють в об'ємі, компенсують один одного всюди, окрім точок бічної
		поверхні. У результаті залишається тільки поверхневий струм, що тече бічною поверхнею циліндра.
	\end{block}
	\begin{columns}
		\begin{column}{0.5\linewidth}\centering
			\begin{tikzpicture}[>=latex]
				\draw[fill=gray!20] (0,0) arc(180:350:1 and 0.2) -- ++(45:2)  arc(350:180:1 and 0.2) -- cycle;
%				\draw[fill=gray!20] (45:2) ++(1,0) coordinate (O) circle(1 and 0.2);
                \path (45:2) ++(1,0) coordinate (O);

				\begin{scope}[shift={(O)}, yscale=0.2]
					\draw[thick, blue, arrowpos={0.7}{2pt}{3pt}] (0,0) [partial ellipse=0:360:1];
					\foreach \a in {10, 40,...,340} {
							\draw[red, smooth, arrowpos={0.75}{2pt}{3pt}] (\a:0.8) [partial ellipse=0:360:0.2];
						}
					\foreach \a in {25,85,...,335} {
							\draw[red, smooth] (\a:0.42) [partial ellipse=0:360:0.2];
						}
					\draw[red, smooth] (0, 0) [partial ellipse=0:360:0.2];
				\end{scope}

				\foreach \l in {0.2,0.4,...,1.8} {
						\draw[blue, arrowpos={0.4}{2pt}{3pt}] (45:\l) arc(180:350:1 and 0.2);
					}
				\draw[->, ultra thick] (O) -- ++(0,1) coordinate (J) node[left] {$\vect{J}$};
				\draw (0,0) -- ++(0,2);
				\draw (0,0) ++(0, 0.5) arc (90:45:0.5) node[pos=0.5, anchor=south] {$\theta$};
                \draw[->] (O) -- ++(45:{1*cos(45)}) coordinate (Pr) node[right] {$\ell\vect{\tau}$};
                \draw[dashed] (J) -- (Pr);
				\draw[<->] (2.2, 0) -- node[right] {$\ell$} ++(45:2);
			\end{tikzpicture}
		\end{column}
		\begin{column}{0.5\linewidth}\centering
			\begin{tikzpicture}[>=latex]
					\draw[arrowpos={0.4}{2pt}{4pt}, thick, blue] (0,0) [partial ellipse=0:360:1.01];
					\foreach[evaluate=\a as \p using \a/360] \a in {10, 40,...,340} {
							\draw[arrowpos={\p}{2pt}{3pt}, red] (\a:0.8) [partial ellipse=0:360:0.2];
						}
					\foreach \a in {25,85,...,335} {
							\draw[arrowpos={\a/360}{2pt}{3pt}, red] (\a:0.42) [partial ellipse=0:360:0.2];
						}
					\draw[arrowpos={0.5}{2pt}{3pt}, red] (0, 0) [partial ellipse=0:360:0.2];
			\end{tikzpicture}
		\end{column}
	\end{columns}

\end{frame}
% ===========================================================================


% ============================== Слайд ## ===================================
\begin{frame}{}{}
\begin{tikzpicture}[]
    \draw (0,0) -- ++(4, 0) -- ++(-45:2) -- ++(-4, 0) --cycle;
    \foreach \x in {0,1,...,8}{
        \draw[arrowpos={0.65}{2pt}{3pt}, red, scale=0.5] (\x, 0) [partial ellipse=10:350:0.2 and 1];
        \draw[arrowpos={0.65}{2pt}{3pt}, red, scale=0.5] ({\x+3}, -2.8) [partial ellipse=10:350:0.2 and 1];
    }
\end{tikzpicture}
\end{frame}
% ===========================================================================


\end{document}
