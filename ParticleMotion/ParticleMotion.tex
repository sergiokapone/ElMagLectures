% !TeX program = lualatex
% !TeX encoding = utf8
% !TeX spellcheck = uk_UA
% !BIB program = bibler

\documentclass[onlytextwidth]{beamer}
\usetheme{Electromagnetism}
\usepackage{Electromagnetism}
\usepackage{circuitikz}


%============================================================================
\title[Лекції електрики та магнетизму]{\huge\bfseries Рух заряджених частинок\\ в електричному та магнітному полях}
\subtitle{Лекції з електрики та магнетизму}
\author{Пономаренко С. М.}
\date{}
%============================================================================
\graphicspath{{pictures/}}
\begin{document}
\begin{frame}[plain]
	\maketitle
\end{frame}

% ============================== Слайд ## ===================================
\begin{frame}{Зміст}{}
	\tableofcontents
\end{frame}
% ===========================================================================


%% --------------------------------------------------------
\section{Рух в однорідному електричному полі}
%% --------------------------------------------------------

% ============================== Слайд ## ===================================
\begin{frame}{Рух в однорідному електричному полі}{}
	\begin{block}{}\justifying
		Якщо напруженість поля $\Efield=\const$, то з рівняння руху:
		\begin{equation*}
			m\dot{\vect{v}} = q\Efield
		\end{equation*}
		випливає:
		\begin{equation*}
			\vect{v} = \vect{v}_0 + \frac{q}{m}\Efield t,\quad \vect{r} = \vect{r}_0 + \vect{v}_0t +   \frac{q}{2m}\Efield t^2,
		\end{equation*}
		тобто має місце рівноприскорений рух із прискоренням, напрямленим уздовж вектора напруженості поля.
	\end{block}
\end{frame}
% ===========================================================================


%% --------------------------------------------------------
\section{Рух в однорідному магнітному полі}
%% --------------------------------------------------------


% ============================== Слайд ## ===================================
\begin{frame}{Рух в однорідному магнітному полі}{}
	\begin{block}{}
		На частинку діє сила Лоренца, так що рівняння руху має вигляд:
		\begin{equation*}
			m\dot{\vect{v}} = q\left[\frac{\vect{v}}{c}\times\Bfield\right].
		\end{equation*}
		Очевидно, що $\dot{\vect{v}}\perp\vect{v}$ і $\dot{\vect{v}}\perp\Bfield$. Розкладемо вектор швидкості на
		складові паралельну і перпендикулярну полю:
		\begin{equation*}
			\vect{v} = \vect{v}_{\perp} + \vect{v}_{\parallel}
		\end{equation*}
		Для цих складових маємо рівняння:
		\begin{equation*}
			\begin{cases}
				m\vect{v}_{\parallel} = 0, \\
				\vect{v}_{\perp} = q\left[\dfrac{\vect{v}_{\perp}}{c}\times\Bfield\right].
			\end{cases}
		\end{equation*}
		З першого рівняння випливає $\vect{v}_{\parallel} = \const$.
Друге рівняння перепишемо у вигляді:
\begin{equation*}
    \vect{v}_{\perp} = \left[\vect{\omega}\times\dfrac{\vect{v}_{\perp}}{c}\right],\quad \tcbhighmath{\vect{\omega} = - \frac{q\Bfield}{mc}.}
\end{equation*}
Рівняння описує обертання навколо напрямку магнітного поля з кутовою швидкістю $\omega$, яка називається \alert{ циклотронною частотою}.
	\end{block}
\end{frame}
% ===========================================================================



%% --------------------------------------------------------
\section{Рух в однорідних паралельних полях \texorpdfstring{$\Efield\parallel\Bfield $}{E||B}}
%% --------------------------------------------------------



% ============================== Слайд ## ===================================
\begin{frame}{Рух в однорідних паралельних полях $\Efield\parallel\Bfield $}{}

\end{frame}
% ===========================================================================



%% --------------------------------------------------------
\section{Рух в однорідних схрещених полях \texorpdfstring{$\Efield\perp\Bfield$}{E⟂B}. Дрейф}
%% --------------------------------------------------------


% ============================== Слайд ## ===================================
\begin{frame}{Рух в однорідних схрещених полях $\Efield\perp\Bfield$. Дрейф}{}

\end{frame}
% ===========================================================================



\end{document}
