
\documentclass{beamer}
\usetheme{Electromagnetism}
\usepackage{Electromagnetism}
\graphicspath{{pictures/}}
% -------------------------------------- Grid
%-------------------------------------------------------
\makeatletter
\def\grd@save@target#1{%
  \def\grd@target{#1}}
\def\grd@save@start#1{%
  \def\grd@start{#1}}
\tikzset{
  grid with coordinates/.style={
    to path={%
      \pgfextra{%
        \edef\grd@@target{(\tikztotarget)}%
        \tikz@scan@one@point\grd@save@target\grd@@target\relax
        \edef\grd@@start{(\tikztostart)}%
        \tikz@scan@one@point\grd@save@start\grd@@start\relax
        \draw[minor help lines] (\tikztostart) grid (\tikztotarget);
        \draw[major help lines] (\tikztostart) grid (\tikztotarget);
        \grd@start
        \pgfmathsetmacro{\grd@xa}{\the\pgf@x/1cm}
        \pgfmathsetmacro{\grd@ya}{\the\pgf@y/1cm}
        \grd@target
        \pgfmathsetmacro{\grd@xb}{\the\pgf@x/1cm}
        \pgfmathsetmacro{\grd@yb}{\the\pgf@y/1cm}
        \pgfmathsetmacro{\grd@xc}{\grd@xa + \pgfkeysvalueof{/tikz/grid with coordinates/major step}}
        \pgfmathsetmacro{\grd@yc}{\grd@ya + \pgfkeysvalueof{/tikz/grid with coordinates/major step}}
        \foreach \x in {\grd@xa,\grd@xc,...,\grd@xb}
        \node[anchor=north] at (\x,\grd@ya) {\pgfmathprintnumber{\x}};
        \foreach \y in {\grd@ya,\grd@yc,...,\grd@yb}
        \node[anchor=east] at (\grd@xa,\y) {\pgfmathprintnumber{\y}};
      }
    }
  },
  minor help lines/.style={
    help lines,
    step=\pgfkeysvalueof{/tikz/grid with coordinates/minor step}
  },
  major help lines/.style={
    help lines,
    line width= 0.5pt,
    step=\pgfkeysvalueof{/tikz/grid with coordinates/major step}
  },
  grid with coordinates/.cd,
  minor step/.initial=.2,
  major step/.initial=1,
  major line width/.initial=2pt,
}
\makeatother
\usepackage{cancel}

\usetikzlibrary{decorations.pathmorphing,
                patterns, positioning,
                quotes,
                shapes.geometric}
\begin{document}



% ============================== Слайд ## ===================================
\begin{frame}{}{}
			\begin{tikzpicture}[scale=0.6, >=latex]
				\clip (-0.5,-0.5) rectangle (6,5);
				\draw[->] (0, -1) to (0,5) node[below left] {$y$};
				\draw[->] (-1, 0) to (6,0) node[below left] {$x$};
				\foreach \i in {0,1,...,5} {
						‎\draw[domain=-1:30, smooth, variable=\x, blue] plot ({\x}, {-\x + \i}) ;
					}
				\coordinate (K) at (1.5, {-1.5 + 4});
                \coordinate (K1) at (0.5, {-0.5 + 3});
                \coordinate (K2) at (0.6, {-0.6 + 4});
                \coordinate (K3) at (0.6, {-0.6 + 5});
				\draw[->] (0,0) -- node[left=1pt] {$\vect{r}$} (K);
				\draw[->, red, double] (K) -- ++(45:1) node[right] (B) {$\k$};

                \node[font=\tiny, align=center] (k) at (4.5, 2) {Хвильовий вектор\\ --- задає напрям \\ поширення хвилі};
                \draw[-stealth, dash pattern={on 1pt off 0.5pt}] (k.90) to[bend right] (B);

                \node[font=\tiny, align=center] (P) at (4,4.5) {Хвильові\\ площини};
                \draw[-stealth, dash pattern={on 1pt off 0.5pt}] (P.180) coordinate (k1) to[] (K1);
                \draw[-stealth, dash pattern={on 1pt off 0.5pt}] (P.180) coordinate (k1) to[] (K2);
                \draw[-stealth, dash pattern={on 1pt off 0.5pt}] (P.180) coordinate (k1) to[] (K3);
			\end{tikzpicture}
\end{frame}
% ===========================================================================

\end{document}