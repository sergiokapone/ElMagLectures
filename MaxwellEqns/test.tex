
\documentclass{beamer}
\usetheme{Electromagnetism}
\usepackage{Electromagnetism}
\graphicspath{{pictures/}}
% -------------------------------------- Grid
%-------------------------------------------------------
\makeatletter
\def\grd@save@target#1{%
  \def\grd@target{#1}}
\def\grd@save@start#1{%
  \def\grd@start{#1}}
\tikzset{
  grid with coordinates/.style={
    to path={%
      \pgfextra{%
        \edef\grd@@target{(\tikztotarget)}%
        \tikz@scan@one@point\grd@save@target\grd@@target\relax
        \edef\grd@@start{(\tikztostart)}%
        \tikz@scan@one@point\grd@save@start\grd@@start\relax
        \draw[minor help lines] (\tikztostart) grid (\tikztotarget);
        \draw[major help lines] (\tikztostart) grid (\tikztotarget);
        \grd@start
        \pgfmathsetmacro{\grd@xa}{\the\pgf@x/1cm}
        \pgfmathsetmacro{\grd@ya}{\the\pgf@y/1cm}
        \grd@target
        \pgfmathsetmacro{\grd@xb}{\the\pgf@x/1cm}
        \pgfmathsetmacro{\grd@yb}{\the\pgf@y/1cm}
        \pgfmathsetmacro{\grd@xc}{\grd@xa + \pgfkeysvalueof{/tikz/grid with coordinates/major step}}
        \pgfmathsetmacro{\grd@yc}{\grd@ya + \pgfkeysvalueof{/tikz/grid with coordinates/major step}}
        \foreach \x in {\grd@xa,\grd@xc,...,\grd@xb}
        \node[anchor=north] at (\x,\grd@ya) {\pgfmathprintnumber{\x}};
        \foreach \y in {\grd@ya,\grd@yc,...,\grd@yb}
        \node[anchor=east] at (\grd@xa,\y) {\pgfmathprintnumber{\y}};
      }
    }
  },
  minor help lines/.style={
    help lines,
    step=\pgfkeysvalueof{/tikz/grid with coordinates/minor step}
  },
  major help lines/.style={
    help lines,
    line width= 0.5pt,
    step=\pgfkeysvalueof{/tikz/grid with coordinates/major step}
  },
  grid with coordinates/.cd,
  minor step/.initial=.2,
  major step/.initial=1,
  major line width/.initial=2pt,
}
\makeatother
\usepackage{cancel}
\usetikzlibrary{shapes.geometric,calc}

\begin{document}


% ============================== Слайд ## ===================================
\begin{frame}{}{}
	\begin{center}
		\begin{tikzpicture}[>=latex]
			\draw [arrowpos={0.9}{2pt}{4pt}, in=-45, out=45, looseness=1.25, line width=1pt, gray!50] (-1, -1) to (-1, 1) node[above, text=black] {$I_1 >
					0$};
			\draw [arrowpos={0.1}{2pt}{4pt}, out=225, looseness=0.80, line width=1pt, gray!50] (0.2, 1) node[above, text=black ] {$I_2 < 0$} to (0.2,
			-1) ;
			\draw [arrowpos={0.1}{2pt}{4pt}, line width=1pt, gray!50]  (0.9, 0.1)  circle(0.2 and 0.7) node[above right=0.4cm, text=black] {$I_3 <
						0$};
			\draw[thick, green!50!black, fill=red!20] (0, 0) circle(1 and 0.3);
			\draw[thick, arrowpos={0.4}{3pt}{4pt}, green!50!black] (-1, 0) arc(180:360:1 and 0.3) node[pos=0.4, below] {$L$};
            \draw[->, red, thick] (0,0) -- ++(0, 0.5);
			\node[text=red] at (0.3, 0) {$S$};
			\begin{scope}
				\clip (-1, 0) rectangle ++(2, 0.5);
				\draw [in=-45, out=45, looseness=1.25, line width=1pt, gray!50] (-1, -1) to (-1, 1);
				\draw [out=225, looseness=0.80, line width=1pt, gray!50] (0.2, 1)  to (0.2, -1);
				\draw [line width=1pt, gray!50]  (0.9, 0.1)  circle(0.2 and 0.7);
			\end{scope}
		\end{tikzpicture}
	\end{center}
\end{frame}
% ===========================================================================



\end{document}