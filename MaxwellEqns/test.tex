
\documentclass{beamer}
\usetheme{Electromagnetism}
\usepackage{Electromagnetism}
\graphicspath{{pictures/}}
% -------------------------------------- Grid
%-------------------------------------------------------
\makeatletter
\def\grd@save@target#1{%
  \def\grd@target{#1}}
\def\grd@save@start#1{%
  \def\grd@start{#1}}
\tikzset{
  grid with coordinates/.style={
    to path={%
      \pgfextra{%
        \edef\grd@@target{(\tikztotarget)}%
        \tikz@scan@one@point\grd@save@target\grd@@target\relax
        \edef\grd@@start{(\tikztostart)}%
        \tikz@scan@one@point\grd@save@start\grd@@start\relax
        \draw[minor help lines] (\tikztostart) grid (\tikztotarget);
        \draw[major help lines] (\tikztostart) grid (\tikztotarget);
        \grd@start
        \pgfmathsetmacro{\grd@xa}{\the\pgf@x/1cm}
        \pgfmathsetmacro{\grd@ya}{\the\pgf@y/1cm}
        \grd@target
        \pgfmathsetmacro{\grd@xb}{\the\pgf@x/1cm}
        \pgfmathsetmacro{\grd@yb}{\the\pgf@y/1cm}
        \pgfmathsetmacro{\grd@xc}{\grd@xa + \pgfkeysvalueof{/tikz/grid with coordinates/major step}}
        \pgfmathsetmacro{\grd@yc}{\grd@ya + \pgfkeysvalueof{/tikz/grid with coordinates/major step}}
        \foreach \x in {\grd@xa,\grd@xc,...,\grd@xb}
        \node[anchor=north] at (\x,\grd@ya) {\pgfmathprintnumber{\x}};
        \foreach \y in {\grd@ya,\grd@yc,...,\grd@yb}
        \node[anchor=east] at (\grd@xa,\y) {\pgfmathprintnumber{\y}};
      }
    }
  },
  minor help lines/.style={
    help lines,
    step=\pgfkeysvalueof{/tikz/grid with coordinates/minor step}
  },
  major help lines/.style={
    help lines,
    line width= 0.5pt,
    step=\pgfkeysvalueof{/tikz/grid with coordinates/major step}
  },
  grid with coordinates/.cd,
  minor step/.initial=.2,
  major step/.initial=1,
  major line width/.initial=2pt,
}
\makeatother
\usepackage{cancel}
\begin{document}


% ============================== Слайд ## ===================================
\begin{frame}{Система рівнянь Максвелла}{}
\begin{block}{}\justifying\small
    Доповнивши основні факти зі сфери електромагнетизму, та доповнивши їх гіпотезою струмів зміщення, Максвелл зміг написати систему фундаментальних
    рівнянь електродинаміки. Таких рівнянь чотири.
\end{block}
       \begin{center}
		\begin{tblr}%
			{
			colspec={|Q[l,m, 2.5cm, bg=yellow!10, font=\scriptsize]|[1pt, white]X[c,m, bg=red!10]|[1pt, white]Q[c,m, 2.5cm, bg=blue!10]|},
			row{1}={c, fg=white, bg=cyan, font=\bfseries\scriptsize},
            cell{2-5}{2-Z}={c, m, mode=dmath, font=\scriptsize},
			%hlines,
			}
			\hline
			Рівняння & Інтегральна форма & Диференціальна форма\\
			\hline\hline
            Теорема Гаусса для магнітного поля         & \oiint\limits_S \Bfield\cdot d\vect{S} =  0 & \Div\Bfield = 0\\
			\hline
            Теорема про циркуляцію для електричного поля & \oint\limits_L \Efield\cdot d\vect{r} = -
			\frac1c\iint\limits_S \frac{\partial\Bfield}{\partial t} \cdot d\vect{S}  &  \Rot\Efield = - \frac1c \frac{\partial\Bfield}{\partial t} \\
            \hline\hline
			Теорема Гаусса для електричного поля         & \oiint\limits_S \Dfield\cdot d\vect{S} = 4\pi \iiint\limits_{V} \rho dV & \Div\Dfield =
			4\pi\rho\\
			\hline
			Теорема про циркуляцію для магнітного поля & \oint\limits_L \Hfield\cdot d\vect{r} = \frac{4\pi}{c} \iint\limits_S \left( \vect{j} +
			\frac1{4\pi}\frac{\partial\Dfield}{\partial t} \right) \cdot d\vect{S} &  \Rot\Hfield = \frac{4\pi}{c}\vect{j} + \frac1c
			\frac{\partial\Dfield}{\partial
			t}
			\\
            \hline
		\end{tblr}
	\end{center}
\end{frame}
% ===========================================================================


\end{document}