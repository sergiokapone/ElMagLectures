
\documentclass{beamer}
\usetheme{Electromagnetism}
\usepackage{Electromagnetism}
\graphicspath{{pictures/}}
% -------------------------------------- Grid
%-------------------------------------------------------
\makeatletter
\def\grd@save@target#1{%
  \def\grd@target{#1}}
\def\grd@save@start#1{%
  \def\grd@start{#1}}
\tikzset{
  grid with coordinates/.style={
    to path={%
      \pgfextra{%
        \edef\grd@@target{(\tikztotarget)}%
        \tikz@scan@one@point\grd@save@target\grd@@target\relax
        \edef\grd@@start{(\tikztostart)}%
        \tikz@scan@one@point\grd@save@start\grd@@start\relax
        \draw[minor help lines] (\tikztostart) grid (\tikztotarget);
        \draw[major help lines] (\tikztostart) grid (\tikztotarget);
        \grd@start
        \pgfmathsetmacro{\grd@xa}{\the\pgf@x/1cm}
        \pgfmathsetmacro{\grd@ya}{\the\pgf@y/1cm}
        \grd@target
        \pgfmathsetmacro{\grd@xb}{\the\pgf@x/1cm}
        \pgfmathsetmacro{\grd@yb}{\the\pgf@y/1cm}
        \pgfmathsetmacro{\grd@xc}{\grd@xa + \pgfkeysvalueof{/tikz/grid with coordinates/major step}}
        \pgfmathsetmacro{\grd@yc}{\grd@ya + \pgfkeysvalueof{/tikz/grid with coordinates/major step}}
        \foreach \x in {\grd@xa,\grd@xc,...,\grd@xb}
        \node[anchor=north] at (\x,\grd@ya) {\pgfmathprintnumber{\x}};
        \foreach \y in {\grd@ya,\grd@yc,...,\grd@yb}
        \node[anchor=east] at (\grd@xa,\y) {\pgfmathprintnumber{\y}};
      }
    }
  },
  minor help lines/.style={
    help lines,
    step=\pgfkeysvalueof{/tikz/grid with coordinates/minor step}
  },
  major help lines/.style={
    help lines,
    line width= 0.5pt,
    step=\pgfkeysvalueof{/tikz/grid with coordinates/major step}
  },
  grid with coordinates/.cd,
  minor step/.initial=.2,
  major step/.initial=1,
  major line width/.initial=2pt,
}
\makeatother
\usepackage{cancel}
\begin{document}



% ============================== Слайд ## ===================================
\begin{frame}{Граничні умови}{}\small
       \begin{block}{}\justifying
    Диференціальні рівняння Максвелла треба доповнити граничними умовами, яким має задовольняти електромагнітне поле на межі
розділу двох середовищ. Ці умови неявно містяться в інтегральній формі рівнянь Максвелла і отримуються за їх допомогою. Граничні умови в стаціонарному
випадку аналогічні і для випадку змінних полів.
\end{block}
       \begin{center}
		\begin{tblr}%
			{
			colspec={|X[l,m, bg=yellow!10, font=\scriptsize]|[1pt, white]X[c,m, bg=red!10]|},
			row{1}={c, fg=white, bg=cyan, font=\bfseries},
            cell{2-Z}{1-Z}={c, m, mode=dmath, font=},
			%hlines,
			}
			\hline
			Умови для електричних векторів & Умови для магнітних векторів \\
			\hline
             D_{2n} - D_{1n} = 4\pi\sigma
             &
             B_{1n} = B_{2n}
            \\
            \hline
			\hline
             E_{1\tau} = E_{2\tau}
             &
             \left[\vect{n}\times\Hfield_2\right] - \left[\vect{n}\times\Hfield_1\right] = \frac{4\pi}{c}\vect{i}
            \\
            \hline
		\end{tblr}
	\end{center}
\begin{block}{}\justifying
    Тут $\sigma$ ---  поверхнева густина вільних електричних зарядів, a $\vect{i}$ --- поверхнева густина струму провідності на розглянутій границі
    розділу. У випадку, коли поверхневих струмів немає, гранична умова для тангенціальної компоненти вектора напруженості магнітного поля набуває
    вигляду:
\begin{equation*}
    H_{1\tau} = H_{2\tau}
\end{equation*}
\end{block}
\end{frame}
% ===========================================================================

\end{document}