
\documentclass{beamer}
\usetheme{Electromagnetism}
\usepackage{Electromagnetism}
\graphicspath{{pictures/}}
% -------------------------------------- Grid
%-------------------------------------------------------
\makeatletter
\def\grd@save@target#1{%
  \def\grd@target{#1}}
\def\grd@save@start#1{%
  \def\grd@start{#1}}
\tikzset{
  grid with coordinates/.style={
    to path={%
      \pgfextra{%
        \edef\grd@@target{(\tikztotarget)}%
        \tikz@scan@one@point\grd@save@target\grd@@target\relax
        \edef\grd@@start{(\tikztostart)}%
        \tikz@scan@one@point\grd@save@start\grd@@start\relax
        \draw[minor help lines] (\tikztostart) grid (\tikztotarget);
        \draw[major help lines] (\tikztostart) grid (\tikztotarget);
        \grd@start
        \pgfmathsetmacro{\grd@xa}{\the\pgf@x/1cm}
        \pgfmathsetmacro{\grd@ya}{\the\pgf@y/1cm}
        \grd@target
        \pgfmathsetmacro{\grd@xb}{\the\pgf@x/1cm}
        \pgfmathsetmacro{\grd@yb}{\the\pgf@y/1cm}
        \pgfmathsetmacro{\grd@xc}{\grd@xa + \pgfkeysvalueof{/tikz/grid with coordinates/major step}}
        \pgfmathsetmacro{\grd@yc}{\grd@ya + \pgfkeysvalueof{/tikz/grid with coordinates/major step}}
        \foreach \x in {\grd@xa,\grd@xc,...,\grd@xb}
        \node[anchor=north] at (\x,\grd@ya) {\pgfmathprintnumber{\x}};
        \foreach \y in {\grd@ya,\grd@yc,...,\grd@yb}
        \node[anchor=east] at (\grd@xa,\y) {\pgfmathprintnumber{\y}};
      }
    }
  },
  minor help lines/.style={
    help lines,
    step=\pgfkeysvalueof{/tikz/grid with coordinates/minor step}
  },
  major help lines/.style={
    help lines,
    line width= 0.5pt,
    step=\pgfkeysvalueof{/tikz/grid with coordinates/major step}
  },
  grid with coordinates/.cd,
  minor step/.initial=.2,
  major step/.initial=1,
  major line width/.initial=2pt,
}
\makeatother
\usepackage{cancel}
\usetikzlibrary{shapes.geometric,calc}


\tikzset{
        custom dash/.style={
            dash pattern=on 2pt off 0.7pt
        }
    }

\begin{document}


% ============================== Слайд ## ===================================
\begin{frame}{}{}
\begin{center}
\begin{tikzpicture}[>=latex]
	\draw [in=-105, out=75, looseness=1.25, line width=0.1cm, gray!50] (-1, -1) coordinate (A) to (1, 1.5) coordinate (B);

	\draw[fill=gray!50, ultra thin, rotate around={-10:(B)}] (B) circle(0.07 and 0.02);
	\draw[fill=gray!50, ultra thin, rotate around={-10:(A)}] (A) circle(0.07 and 0.02);
    % Bfield

    \draw[thin, blue, rotate around={-15:(-0.25:0)}, arrowpos={0.85}{2pt}{2pt}] (-0.25, 0) circle(0.5 and 0.15);
    \draw[thin, blue, rotate around={-15:(-0.25:0)}, arrowpos={0.85}{2pt}{2pt}] (-0.25, 0) circle(2 and 0.7);
	\draw[thin, blue, rotate around={-15:(-0.25:0)}, arrowpos={0.85}{2pt}{2pt}] (-0.25, 0) circle(1 and 0.3);
	\begin{scope}
		\clip (-1,0) rectangle (1,1);
		\draw [in=-105, out=75, looseness=1.25, line width=0.1cm, gray!50] (A) to (B) ;
	\end{scope}
    \draw [in=-105, out=75, looseness=1.25, gray, arrowpos={0.85}{2pt}{4pt}] (-1, -1) coordinate (A) to (1, 1.5) coordinate (B);
\end{tikzpicture}
\end{center}
\end{frame}
% ===========================================================================



\end{document}