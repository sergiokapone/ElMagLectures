
\documentclass{beamer}
\usetheme{Electromagnetism}
\usepackage{Electromagnetism}
\graphicspath{{pictures/}}
% -------------------------------------- Grid
%-------------------------------------------------------
\makeatletter
\def\grd@save@target#1{%
  \def\grd@target{#1}}
\def\grd@save@start#1{%
  \def\grd@start{#1}}
\tikzset{
  grid with coordinates/.style={
    to path={%
      \pgfextra{%
        \edef\grd@@target{(\tikztotarget)}%
        \tikz@scan@one@point\grd@save@target\grd@@target\relax
        \edef\grd@@start{(\tikztostart)}%
        \tikz@scan@one@point\grd@save@start\grd@@start\relax
        \draw[minor help lines] (\tikztostart) grid (\tikztotarget);
        \draw[major help lines] (\tikztostart) grid (\tikztotarget);
        \grd@start
        \pgfmathsetmacro{\grd@xa}{\the\pgf@x/1cm}
        \pgfmathsetmacro{\grd@ya}{\the\pgf@y/1cm}
        \grd@target
        \pgfmathsetmacro{\grd@xb}{\the\pgf@x/1cm}
        \pgfmathsetmacro{\grd@yb}{\the\pgf@y/1cm}
        \pgfmathsetmacro{\grd@xc}{\grd@xa + \pgfkeysvalueof{/tikz/grid with coordinates/major step}}
        \pgfmathsetmacro{\grd@yc}{\grd@ya + \pgfkeysvalueof{/tikz/grid with coordinates/major step}}
        \foreach \x in {\grd@xa,\grd@xc,...,\grd@xb}
        \node[anchor=north] at (\x,\grd@ya) {\pgfmathprintnumber{\x}};
        \foreach \y in {\grd@ya,\grd@yc,...,\grd@yb}
        \node[anchor=east] at (\grd@xa,\y) {\pgfmathprintnumber{\y}};
      }
    }
  },
  minor help lines/.style={
    help lines,
    step=\pgfkeysvalueof{/tikz/grid with coordinates/minor step}
  },
  major help lines/.style={
    help lines,
    line width= 0.5pt,
    step=\pgfkeysvalueof{/tikz/grid with coordinates/major step}
  },
  grid with coordinates/.cd,
  minor step/.initial=.2,
  major step/.initial=1,
  major line width/.initial=2pt,
}
\makeatother
\usepackage{cancel}
\usetikzlibrary{shapes.geometric,calc}

\begin{document}


% ============================== Слайд ## ===================================
\begin{frame}{}{}
	\begin{center}
			\begin{tikzpicture}[>=latex, scale=0.7, transform shape]
				\draw [in=-105, out=75, looseness=1.25, line width=12pt, gray!50] (-1, -1) coordinate (A) to (2, 2) coordinate (B);
				\draw[fill=gray!50, ultra thin, rotate around={-10:(B)}] (B) circle(8.5pt and 2pt);
				\draw[fill=gray!50, ultra thin, rotate around={-10:(A)}] (A) circle(8.5pt and 2pt);
				\draw[thick, green!50!black] (0, 0) circle(1 and 0.3);
				\begin{scope}
					\clip (-1,0) rectangle (1,1);
					\draw [in=-105, out=75, looseness=1.25, line width=12pt, gray!50] (-1, -1) to (2, 2) ;
				\end{scope}
				\draw[red!20, pattern=crosshatch, pattern color=red!20] (-1,0) to[out=85, in=45, looseness=3] node[pos=0.2, above, text=red]
				{$S$} (1, 0) arc(0:-180:1 and 0.3) ;
				\draw[thick, arrowpos={0.5}{3pt}{4pt}, green!50!black] (-1, 0) arc(180:360:1 and 0.3) node[pos=0.5, below] {$L$};
			\end{tikzpicture}
	\end{center}
\end{frame}
% ===========================================================================



\end{document}