
\documentclass{beamer}
\usetheme{Electromagnetism}
\usepackage{Electromagnetism}
\graphicspath{{pictures/}}
% -------------------------------------- Grid
%-------------------------------------------------------
\makeatletter
\def\grd@save@target#1{%
  \def\grd@target{#1}}
\def\grd@save@start#1{%
  \def\grd@start{#1}}
\tikzset{
  grid with coordinates/.style={
    to path={%
      \pgfextra{%
        \edef\grd@@target{(\tikztotarget)}%
        \tikz@scan@one@point\grd@save@target\grd@@target\relax
        \edef\grd@@start{(\tikztostart)}%
        \tikz@scan@one@point\grd@save@start\grd@@start\relax
        \draw[minor help lines] (\tikztostart) grid (\tikztotarget);
        \draw[major help lines] (\tikztostart) grid (\tikztotarget);
        \grd@start
        \pgfmathsetmacro{\grd@xa}{\the\pgf@x/1cm}
        \pgfmathsetmacro{\grd@ya}{\the\pgf@y/1cm}
        \grd@target
        \pgfmathsetmacro{\grd@xb}{\the\pgf@x/1cm}
        \pgfmathsetmacro{\grd@yb}{\the\pgf@y/1cm}
        \pgfmathsetmacro{\grd@xc}{\grd@xa + \pgfkeysvalueof{/tikz/grid with coordinates/major step}}
        \pgfmathsetmacro{\grd@yc}{\grd@ya + \pgfkeysvalueof{/tikz/grid with coordinates/major step}}
        \foreach \x in {\grd@xa,\grd@xc,...,\grd@xb}
        \node[anchor=north] at (\x,\grd@ya) {\pgfmathprintnumber{\x}};
        \foreach \y in {\grd@ya,\grd@yc,...,\grd@yb}
        \node[anchor=east] at (\grd@xa,\y) {\pgfmathprintnumber{\y}};
      }
    }
  },
  minor help lines/.style={
    help lines,
    step=\pgfkeysvalueof{/tikz/grid with coordinates/minor step}
  },
  major help lines/.style={
    help lines,
    line width= 0.5pt,
    step=\pgfkeysvalueof{/tikz/grid with coordinates/major step}
  },
  grid with coordinates/.cd,
  minor step/.initial=.2,
  major step/.initial=1,
  major line width/.initial=2pt,
}
\makeatother
\usepackage{cancel}
\usetikzlibrary{shapes.geometric,calc}

\begin{document}


% ============================== Слайд ## ===================================
\begin{frame}{}{}
	\begin{center}
			\begin{tikzpicture}[>=latex, scale=0.7, transform shape]
                \clip (-2.25, -1.2) -- (2.2,4.1);
				\draw [in=-105, out=75, looseness=1.25, line width=0.1cm, gray!50] (-1, -1) coordinate (A) to (1, 1.5) coordinate (B);

				\draw[fill=gray!50, ultra thin, rotate around={-10:(B)}] (B) circle(1cm and 0.3cm);
				\draw[fill=gray!50, ultra thin, rotate around={-10:(A)}] (A) circle(0.07 and 0.02);
				\draw[thick, green!50!black, rotate around={-15:(-0.25:0)}] (-0.25, 0) circle(1 and 0.3);
                % Bfield

                \draw[thin, blue, rotate around={-15:(-0.25:0)}, arrowpos={0.85}{2pt}{2pt}] (-0.25, 0) circle(0.5 and 0.15);
                \draw[thin, blue, rotate around={-15:(-0.25:0)}, arrowpos={0.85}{2pt}{2pt}] (-0.25, 0) circle(2 and 0.7);

				\begin{scope}
					\clip (-1,0) rectangle (1,1);
					\draw [in=-105, out=75, looseness=1.25, line width=0.1cm, gray!50] (A) to (B) ;
				\end{scope}
                \draw [in=-105, out=75, looseness=1.25, gray, , arrowpos={0.2}{2pt}{3pt}] (A) to (B) ;
				\draw[red!20, pattern=crosshatch, pattern color=red!20, rotate around={-15:(-1:0)}] (-1.25, 0) to[out=80, in=40, looseness=4.8]
				node[pos=0.1,
				above, text=red]
				{$S$} (0.75, 0) arc(0:-180:1 and 0.3) -- cycle;
				\draw[thick, arrowpos={0.5}{3pt}{4pt}, green!50!black, rotate around={-15:(-1:0)}] (-1.25, 0) arc(180:360:1 and 0.3) node[pos=0.5,
				below] {$L$};
                \coordinate (C) at (1.1, 2.2);
                \coordinate (D) at (2.1, 4);
                \begin{scope}[shift={(1.2, 1.2)}, rotate around={-10:(1.2, 1.2)}]
                    \foreach \x in {-0.9,-0.7,...,0.9} {
                                        \draw[red, arrowpos={0.4}{2pt}{2pt}] (\x, 0) -- ++(0, 0.7);
                                    }
                \end{scope}
                \draw[fill=gray!50, ultra thin, rotate around={-10:(C)}] (C) circle(1cm and 0.3cm);
                \draw [in=-105, out=85, looseness=1.25, line width=0.1cm, gray!60] (C) to (D);
                \draw [in=-105, out=85, looseness=1.25, arrowpos={0.4}{2pt}{3pt}, gray] (C) to (D);
                \draw[ultra  thin, fill=gray!60, rotate around={-10:(C)}] (C) ++(180:0.07) arc(180:360:0.07 and 0.02);
                \draw[ultra  thin, fill=gray!60, rotate around={-10:(D)}] (D) circle(0.07 and 0.02);

%                \draw (-2, -2) to[grid with coordinates] (3,3);
			\end{tikzpicture}
	\end{center}
\end{frame}
% ===========================================================================



\end{document}