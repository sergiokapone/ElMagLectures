
\documentclass{beamer}
\usetheme{Electromagnetism}
\usepackage{Electromagnetism}
\graphicspath{{pictures/}}
% -------------------------------------- Grid
%-------------------------------------------------------
\makeatletter
\def\grd@save@target#1{%
  \def\grd@target{#1}}
\def\grd@save@start#1{%
  \def\grd@start{#1}}
\tikzset{
  grid with coordinates/.style={
    to path={%
      \pgfextra{%
        \edef\grd@@target{(\tikztotarget)}%
        \tikz@scan@one@point\grd@save@target\grd@@target\relax
        \edef\grd@@start{(\tikztostart)}%
        \tikz@scan@one@point\grd@save@start\grd@@start\relax
        \draw[minor help lines] (\tikztostart) grid (\tikztotarget);
        \draw[major help lines] (\tikztostart) grid (\tikztotarget);
        \grd@start
        \pgfmathsetmacro{\grd@xa}{\the\pgf@x/1cm}
        \pgfmathsetmacro{\grd@ya}{\the\pgf@y/1cm}
        \grd@target
        \pgfmathsetmacro{\grd@xb}{\the\pgf@x/1cm}
        \pgfmathsetmacro{\grd@yb}{\the\pgf@y/1cm}
        \pgfmathsetmacro{\grd@xc}{\grd@xa + \pgfkeysvalueof{/tikz/grid with coordinates/major step}}
        \pgfmathsetmacro{\grd@yc}{\grd@ya + \pgfkeysvalueof{/tikz/grid with coordinates/major step}}
        \foreach \x in {\grd@xa,\grd@xc,...,\grd@xb}
        \node[anchor=north] at (\x,\grd@ya) {\pgfmathprintnumber{\x}};
        \foreach \y in {\grd@ya,\grd@yc,...,\grd@yb}
        \node[anchor=east] at (\grd@xa,\y) {\pgfmathprintnumber{\y}};
      }
    }
  },
  minor help lines/.style={
    help lines,
    step=\pgfkeysvalueof{/tikz/grid with coordinates/minor step}
  },
  major help lines/.style={
    help lines,
    line width= 0.5pt,
    step=\pgfkeysvalueof{/tikz/grid with coordinates/major step}
  },
  grid with coordinates/.cd,
  minor step/.initial=.2,
  major step/.initial=1,
  major line width/.initial=2pt,
}
\makeatother
\usepackage{cancel}
\begin{document}


% ============================== Слайд ## ===================================
\begin{frame}{Довгий провідник}{}\small
	\begin{columns}
		\begin{column}{0.3\linewidth}\centering
			\begin{tikzpicture}[>=latex]
				\draw[fill=gray!50] (0, 0) -- ++(0,2) arc(180:360:1 and 0.3) -- ++(0, -2) arc(0:-180:1 and 0.3);
				\draw[fill=gray!50] (1, 2) circle (1 and 0.3);
				\draw[blue, arrowpos={0.5}{2pt}{4pt}] (1, 1) [partial ellipse=155:385:1.1 and 0.3] node[pos=0.5, below] {$\Hfield$};
				\draw[->, red] (1, 1) -- ++(0, 0.5) node[left] {$\Efield$};
                \draw[->, orange!60!black, thick] (2.1, 1) -- ++(-0.5, 0) node[above] {$\vect{\Pi}$};
			\end{tikzpicture}
		\end{column}
		\begin{column}{0.7\linewidth}
			\begin{block}{}\justifying
				Розглянемо довгий прямий дріт радіуса $R$ і довжини $\ell$, яким тече струм $I$ . На зовнішній поверхні дроту
				присутнє магнітне поле $H = \frac{2I}{cR}$. Струм у дроті виникає
				завдяки напрузі $U$ на його кінцях, що створює поле величиною $E = \frac{U}{\ell}$. Напрямки полів $\Efield$ і $\Hfield$ показано на рис.
			\end{block}
		\end{column}
	\end{columns}
	\begin{block}{}\justifying
		Вектор Пойнтінга виявляється спрямованим до осі дроту і рівнй за величиною:
		\begin{equation*}
			\Pi = \frac{c}{4\pi} EH = \frac{c}{4\pi} \frac{U}{\ell} \frac{2I}{cR} = \frac{1}{2\pi\ell R} UI
		\end{equation*}
		Оскільки площа бічної поверхні дроту дорівнює $S= 2\pi\ell R$, то повний потік енергії, що втікає в дріт, становить $\Pi S = UI$. Це в точності та сама
		енергія, що йде на джоулеві втрати в дроті: $\Pi S = Q = UI$.
	\end{block}
	\begin{block}{}\justifying
		Таким чином, електрони отримують свою енергію, ззовні, від потоку енергії зовнішнього поля всередину дроту і витрачають її на
		створення теплоти. Енергія віддалених зарядів якимось чином розтікається по великій області простору і потім втікає всередину дроту.
	\end{block}
\end{frame}
% ===========================================================================


\end{document}