
\documentclass{beamer}
\usetheme{Electromagnetism}
\usepackage{Electromagnetism}
\graphicspath{{pictures/}}
% -------------------------------------- Grid
%-------------------------------------------------------
\makeatletter
\def\grd@save@target#1{%
  \def\grd@target{#1}}
\def\grd@save@start#1{%
  \def\grd@start{#1}}
\tikzset{
  grid with coordinates/.style={
    to path={%
      \pgfextra{%
        \edef\grd@@target{(\tikztotarget)}%
        \tikz@scan@one@point\grd@save@target\grd@@target\relax
        \edef\grd@@start{(\tikztostart)}%
        \tikz@scan@one@point\grd@save@start\grd@@start\relax
        \draw[minor help lines, gray!50] (\tikztostart) grid (\tikztotarget);
        \draw[major help lines, gray!50] (\tikztostart) grid (\tikztotarget);
        \grd@start
        \pgfmathsetmacro{\grd@xa}{\the\pgf@x/1cm}
        \pgfmathsetmacro{\grd@ya}{\the\pgf@y/1cm}
        \grd@target
        \pgfmathsetmacro{\grd@xb}{\the\pgf@x/1cm}
        \pgfmathsetmacro{\grd@yb}{\the\pgf@y/1cm}
        \pgfmathsetmacro{\grd@xc}{\grd@xa + \pgfkeysvalueof{/tikz/grid with coordinates/major step}}
        \pgfmathsetmacro{\grd@yc}{\grd@ya + \pgfkeysvalueof{/tikz/grid with coordinates/major step}}
        \foreach \x in {\grd@xa,\grd@xc,...,\grd@xb}
        \node[anchor=north] at (\x,\grd@ya) {\pgfmathprintnumber{\x}};
        \foreach \y in {\grd@ya,\grd@yc,...,\grd@yb}
        \node[anchor=east] at (\grd@xa,\y) {\pgfmathprintnumber{\y}};
      }
    }
  },
  minor help lines/.style={
    help lines,
    step=\pgfkeysvalueof{/tikz/grid with coordinates/minor step}
  },
  major help lines/.style={
    help lines,
    line width= 0.5pt,
    step=\pgfkeysvalueof{/tikz/grid with coordinates/major step}
  },
  grid with coordinates/.cd,
  minor step/.initial=.2,
  major step/.initial=1,
  major line width/.initial=2pt,
}
\makeatother

\begin{document}


% ============================== Слайд ## ===================================
\begin{frame}{Густини електричного заряду}{}

\begin{center}
\begin{tikzpicture}[>=latex]
    \draw[gray!60, line width=0.21cm] (0,0)  [partial ellipse=90:0:2];
    \fill[gray!50, draw = gray!70] (0, 2) circle (0.05 and 0.1);
    \fill[gray!50, rotate=-90, draw = gray!70] (0, 2) circle (0.05 and 0.1);


    \foreach \n in {1,...,3} {
        \draw[rotate = -40, blue!40] (0, 2) [partial ellipse={10-2*\n}:{350+2*\n}:{0.2*\n} and \n];
    }

    \node[circle, fill, inner sep=0.5pt] (P) at (3.04,3) {};
    \node[below] at (P) {$P$};

    \draw[red!40, line width=0.21cm] (0,0)  [partial ellipse=55:45:2];
    \fill[red!40, draw = red!60, rotate=-45] (0, 2) circle (0.05 and 0.1);
    \fill[red!40, draw = red!60, rotate=-35] (0, 2) circle (0.05 and 0.1);
    \draw[-{Latex[scale=0.5]}] (55:2)  node[below=1pt] {$\vect{j}$} -- ++({55-90}:0.2);
    \draw[-{Latex[scale=0.5]}] (50:2.1)  -- node[anchor=south east, inner sep=1pt] {$\vect{r}$} (P);
\end{tikzpicture}
\end{center}
\end{frame}
% ===========================================================================





\end{document}