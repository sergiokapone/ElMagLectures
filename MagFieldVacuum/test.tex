
\documentclass{beamer}
\usetheme{Electromagnetism}
\usepackage{Electromagnetism}
\graphicspath{{pictures/}}
% -------------------------------------- Grid
%-------------------------------------------------------
\makeatletter
\def\grd@save@target#1{%
  \def\grd@target{#1}}
\def\grd@save@start#1{%
  \def\grd@start{#1}}
\tikzset{
  grid with coordinates/.style={
    to path={%
      \pgfextra{%
        \edef\grd@@target{(\tikztotarget)}%
        \tikz@scan@one@point\grd@save@target\grd@@target\relax
        \edef\grd@@start{(\tikztostart)}%
        \tikz@scan@one@point\grd@save@start\grd@@start\relax
        \draw[minor help lines, gray!50] (\tikztostart) grid (\tikztotarget);
        \draw[major help lines, gray!50] (\tikztostart) grid (\tikztotarget);
        \grd@start
        \pgfmathsetmacro{\grd@xa}{\the\pgf@x/1cm}
        \pgfmathsetmacro{\grd@ya}{\the\pgf@y/1cm}
        \grd@target
        \pgfmathsetmacro{\grd@xb}{\the\pgf@x/1cm}
        \pgfmathsetmacro{\grd@yb}{\the\pgf@y/1cm}
        \pgfmathsetmacro{\grd@xc}{\grd@xa + \pgfkeysvalueof{/tikz/grid with coordinates/major step}}
        \pgfmathsetmacro{\grd@yc}{\grd@ya + \pgfkeysvalueof{/tikz/grid with coordinates/major step}}
        \foreach \x in {\grd@xa,\grd@xc,...,\grd@xb}
        \node[anchor=north] at (\x,\grd@ya) {\pgfmathprintnumber{\x}};
        \foreach \y in {\grd@ya,\grd@yc,...,\grd@yb}
        \node[anchor=east] at (\grd@xa,\y) {\pgfmathprintnumber{\y}};
      }
    }
  },
  minor help lines/.style={
    help lines,
    step=\pgfkeysvalueof{/tikz/grid with coordinates/minor step}
  },
  major help lines/.style={
    help lines,
    line width= 0.5pt,
    step=\pgfkeysvalueof{/tikz/grid with coordinates/major step}
  },
  grid with coordinates/.cd,
  minor step/.initial=.2,
  major step/.initial=1,
  major line width/.initial=2pt,
}
\makeatother

\begin{document}


% ============================== Слайд ## ===================================
\begin{frame}{Потенціальна енергія диполя в магнітному полі}{}\small
\begin{block}{}\justifying
Розглянемо виток площею $S$, у якому циркулює струм $I$, магнітний момент якого $\vect{p}_m = \frac1c IS\vec{n}$. Вважаємо, що магнітний момент не
змінюється за величиною, тільки може змінювати напрямок у просторі. Останнє припущення істотне, і воно передбачає, що в коло витка ввімкнене
\alert{джерело енергії (ЕРС)}, що підтримує струм незмінним. Якщо виток перебуває в магнітному полі, то виникає момент сил, які прагнуть орієнтувати
його магнітний момент за напрямком поля:
\begin{equation*}
    \vect{M} = \left[\vect{p}_m\times\Bfield\right]
\end{equation*}
З визначення потенціальної енергії знаходимо
\begin{equation*}
    U = -\vect{p}_m\cdot\Bfield
\end{equation*}
\end{block}
       \begin{block}{}
     Той факт, що потенціальна енергія досягає мінімуму $\vect{p}_m\uparrow\uparrow\Bfield$, означає, що момент прагне орієнтуватися за напрямом поля.
\end{block}
\end{frame}
% ===========================================================================


% ============================== Слайд ## ===================================
\begin{frame}{Сила, що діє на диполь в магнітному полі}{}\small
\begin{block}{}\justifying
У зовнішньому магнітному полі потенціальна енергія магнітного моменту дорівнює $U = -\vect{p}_m\cdot\Bfield$, а сила, що діє на момент:
\begin{equation*}
    \vect{F} = -\grad U = \grad(\vect{p}_m\cdot\Bfield).
\end{equation*}
{\scriptsize
\begin{equation*}
    \grad\scdot{\vect{A}}{\vect{B}}  = \vecdot{\vect{B}}{\Rot\vect{A}} + \vecdot{\vect{A}}{\Rot\vect{B}} + \scdot{\vect{B}}{\grad}\vect{A} +
    \scdot{\vect{A}}{\grad}\vect{B}.
\end{equation*}
}
Якщо в середовищі, в якому перебуває момент, відсутні струми провідності, то $\Rot\Bfield = 0$. Тоді має місце тотожність:
\begin{equation*}
    \tcbhighmath{\vect{F} = \scdot{\vect{p}_m}{\grad}\vect{B}.}
\end{equation*}
\end{block}
\begin{block}{}\justifying\scriptsize
У окремому випадку, коли момент спрямований уздовж поля $\vect{p}_m\uparrow\uparrow\vect{B}$, а поле залежить тільки від координати $z$, сила
спрямована по осі $z$ і дорівнює:
\begin{equation*}
    F_z = p_m\frac{dB}{dz}.
\end{equation*}
\end{block}

\href{https://www.youtube.com/watch?v=66yHf8Mv1SQ}{\scriptsize\color{blue} Порівняйте з випадком електричного диполя в неоднорідному електричному
полі}
\end{frame}
% ===========================================================================
% https://tikz.net/magnetic_moment/
\end{document}