
\documentclass{beamer}
\usetheme{Electromagnetism}
\usepackage{Electromagnetism}
\graphicspath{{pictures/}}
% -------------------------------------- Grid
%-------------------------------------------------------
\makeatletter
\def\grd@save@target#1{%
  \def\grd@target{#1}}
\def\grd@save@start#1{%
  \def\grd@start{#1}}
\tikzset{
  grid with coordinates/.style={
    to path={%
      \pgfextra{%
        \edef\grd@@target{(\tikztotarget)}%
        \tikz@scan@one@point\grd@save@target\grd@@target\relax
        \edef\grd@@start{(\tikztostart)}%
        \tikz@scan@one@point\grd@save@start\grd@@start\relax
        \draw[minor help lines, gray!50] (\tikztostart) grid (\tikztotarget);
        \draw[major help lines, gray!50] (\tikztostart) grid (\tikztotarget);
        \grd@start
        \pgfmathsetmacro{\grd@xa}{\the\pgf@x/1cm}
        \pgfmathsetmacro{\grd@ya}{\the\pgf@y/1cm}
        \grd@target
        \pgfmathsetmacro{\grd@xb}{\the\pgf@x/1cm}
        \pgfmathsetmacro{\grd@yb}{\the\pgf@y/1cm}
        \pgfmathsetmacro{\grd@xc}{\grd@xa + \pgfkeysvalueof{/tikz/grid with coordinates/major step}}
        \pgfmathsetmacro{\grd@yc}{\grd@ya + \pgfkeysvalueof{/tikz/grid with coordinates/major step}}
        \foreach \x in {\grd@xa,\grd@xc,...,\grd@xb}
        \node[anchor=north] at (\x,\grd@ya) {\pgfmathprintnumber{\x}};
        \foreach \y in {\grd@ya,\grd@yc,...,\grd@yb}
        \node[anchor=east] at (\grd@xa,\y) {\pgfmathprintnumber{\y}};
      }
    }
  },
  minor help lines/.style={
    help lines,
    step=\pgfkeysvalueof{/tikz/grid with coordinates/minor step}
  },
  major help lines/.style={
    help lines,
    line width= 0.5pt,
    step=\pgfkeysvalueof{/tikz/grid with coordinates/major step}
  },
  grid with coordinates/.cd,
  minor step/.initial=.2,
  major step/.initial=1,
  major line width/.initial=2pt,
}
\makeatother

\begin{document}





% ============================== Слайд ## ===================================
\begin{frame}{Сила, що діє на диполь в магнітному полі}{}\small
	\begin{block}{}\justifying
		У зовнішньому магнітному полі потенціальна енергія магнітного моменту дорівнює $U = -\vect{p}_m\cdot\Bfield$, а сила, що діє на
		момент:
		\begin{equation*}
			\vect{F} = -\grad U = \grad(\vect{p}_m\cdot\Bfield).
		\end{equation*}
	\end{block}
	\begin{columns}
		\begin{column}{0.75\linewidth}
			\begin{block}{}\justifying
				У зовнішньому магнітному полі потенціальна енергія магнітного моменту дорівнює $U = -\vect{p}_m\cdot\Bfield$, а сила, що діє на момент:
				\begin{equation*}
					\vect{F} = -\grad U = \grad(\vect{p}_m\cdot\Bfield).
				\end{equation*}
				{\scriptsize \begin{equation*}
					\grad\scdot{\vect{A}}{\vect{B}}  = \vecdot{\vect{B}}{\Rot\vect{A}} + \vecdot{\vect{A}}{\Rot\vect{B}} +
					\scdot{\vect{B}}{\grad}\vect{A} +
					\scdot{\vect{A}}{\grad}\vect{B}.
				\end{equation*}}

		Якщо в середовищі, в якому перебуває момент, відсутні струми провідності, то $\Rot\Bfield = 0$. Тоді має місце тотожність:
		\begin{equation*}
			\tcbhighmath{\vect{F} = \scdot{\vect{p}_m}{\grad}\vect{B}.}
		\end{equation*}
			\end{block}
		\end{column}
		\begin{column}{0.25\linewidth}\centering
			\begin{tikzpicture}[scale=0.9, >=latex, midarrow/.style={%
							postaction={ decorate,
									decoration={ markings, mark=at position .7 with {\arrow{latex}}}}}]

				\foreach \y in {-3,...,3}{
						\draw[blue, midarrow] plot[domain=1:4] (\x, 0.2*\y+0.05*\y*\x^2);
					}

				\coordinate (AP) at (3, 0);
				\pic[scale=0.7, rotate=210] at (AP) {magarrow};
				\draw[->, ultra thick] (AP) -- ++(-1, 0) node[below=2pt, fill=white, inner sep=1pt] {$\vect{F}$};
			\end{tikzpicture}
		\end{column}
	\end{columns}
%	\begin{block}{}\justifying\scriptsize
%		У окремому випадку, коли момент спрямований уздовж поля $\vect{p}_m\uparrow\uparrow\vect{B}$, а поле залежить тільки від координати
%		$z$, сила
%		спрямована по осі $z$ і дорівнює:
%		\begin{equation*}
%			F_z = p_m\frac{dB}{dz}.
%		\end{equation*}
%	\end{block}
%	\href{https://www.youtube.com/watch?v=66yHf8Mv1SQ}{\scriptsize\color{blue} Порівняйте з випадком електричного диполя в неоднорідному електричному
%		полі}
\end{frame}
% ===========================================================================
% https://tikz.net/magnetic_moment/




\end{document}