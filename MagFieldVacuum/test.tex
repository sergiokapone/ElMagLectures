
\documentclass{beamer}
\usetheme{Electromagnetism}
\usepackage{Electromagnetism}
\graphicspath{{pictures/}}
\usetikzlibrary{spy}
% -------------------------------------- Grid
%-------------------------------------------------------
\makeatletter
\def\grd@save@target#1{%
  \def\grd@target{#1}}
\def\grd@save@start#1{%
  \def\grd@start{#1}}
\tikzset{
  grid with coordinates/.style={
    to path={%
      \pgfextra{%
        \edef\grd@@target{(\tikztotarget)}%
        \tikz@scan@one@point\grd@save@target\grd@@target\relax
        \edef\grd@@start{(\tikztostart)}%
        \tikz@scan@one@point\grd@save@start\grd@@start\relax
        \draw[minor help lines, gray!50] (\tikztostart) grid (\tikztotarget);
        \draw[major help lines, gray!50] (\tikztostart) grid (\tikztotarget);
        \grd@start
        \pgfmathsetmacro{\grd@xa}{\the\pgf@x/1cm}
        \pgfmathsetmacro{\grd@ya}{\the\pgf@y/1cm}
        \grd@target
        \pgfmathsetmacro{\grd@xb}{\the\pgf@x/1cm}
        \pgfmathsetmacro{\grd@yb}{\the\pgf@y/1cm}
        \pgfmathsetmacro{\grd@xc}{\grd@xa + \pgfkeysvalueof{/tikz/grid with coordinates/major step}}
        \pgfmathsetmacro{\grd@yc}{\grd@ya + \pgfkeysvalueof{/tikz/grid with coordinates/major step}}
        \foreach \x in {\grd@xa,\grd@xc,...,\grd@xb}
        \node[anchor=north] at (\x,\grd@ya) {\pgfmathprintnumber{\x}};
        \foreach \y in {\grd@ya,\grd@yc,...,\grd@yb}
        \node[anchor=east] at (\grd@xa,\y) {\pgfmathprintnumber{\y}};
      }
    }
  },
  minor help lines/.style={
    help lines,
    step=\pgfkeysvalueof{/tikz/grid with coordinates/minor step}
  },
  major help lines/.style={
    help lines,
    line width= 0.5pt,
    step=\pgfkeysvalueof{/tikz/grid with coordinates/major step}
  },
  grid with coordinates/.cd,
  minor step/.initial=.2,
  major step/.initial=1,
  major line width/.initial=2pt,
}
\makeatother

\begin{document}





% ============================== Слайд ## ===================================
\begin{frame}{Рух зарядженої частинки в магнітному полі}{}
\framesubtitle{Циклотронна частота}
	\begin{block}{}
		На частинку діє сила Лоренца, так що рівняння руху має вигляд:
		\begin{equation*}
			m\dot{\vect{v}} = q\left[\frac{\vect{v}}{c}\times\Bfield\right].
		\end{equation*}
		Очевидно, що $\dot{\vect{v}}\perp\vect{v}$ і $\dot{\vect{v}}\perp\Bfield$. Розкладемо вектор швидкості на
		складові паралельну і перпендикулярну полю:
		\begin{equation*}
			\vect{v} = \vect{v}_{\perp} + \vect{v}_{\parallel}
		\end{equation*}
		Для цих складових маємо рівняння:
		\begin{equation*}
			\begin{cases}
				m\vect{v}_{\parallel} = 0, \\
				\vect{v}_{\perp} = q\left[\dfrac{\vect{v}_{\perp}}{c}\times\Bfield\right].
			\end{cases}
		\end{equation*}
		З першого рівняння випливає $\vect{v}_{\parallel} = \const$.
Друге рівняння перепишемо у вигляді:
\begin{equation*}
    \vect{v}_{\perp} = \left[\vect{\omega}\times\dfrac{\vect{v}_{\perp}}{c}\right],\quad \tcbhighmath{\vect{\omega} = - \frac{q\Bfield}{mc}.}
\end{equation*}
Рівняння описує обертання навколо напрямку магнітного поля з кутовою швидкістю $\omega$, яка називається \alert{ циклотронною частотою}.
	\end{block}
\end{frame}
% ===========================================================================

% ============================== Слайд ## ===================================
\begin{frame}{Робота магнітного поля}{}
	Робота сили Лоренца:
	\begin{columns}
		\begin{column}{0.65\linewidth}
			\begin{equation*}
				\delta A = \vect{F}\cdot d\vect{r} = \vect{F}\cdot\vect{v}dt = \left[\frac{q \vect{v}}{c}\times\Bfield\right] \cdot\vect{v}dt.
			\end{equation*}
			\begin{block}{}
				\begin{equation*}
					\vect{A}\cdot\vecdot{\vect{B}}{\vect{C}} = \vect{C}\cdot\vecdot{\vect{A}}{\vect{B}}.
				\end{equation*}
			\end{block}
			\begin{equation*}
				\vect{v}\cdot\left[\vect{v}\times\Bfield\right] = \Bfield\cdot\left[\vect{v}\times\vect{v}\right] = 0.
			\end{equation*}
			\begin{equation*}
				\delta A = 0.
			\end{equation*}
		\end{column}
		\begin{column}{0.35\linewidth}\centering
			\begin{tikzpicture}[>=latex, every node/.style={font=\scriptsize}, scale=1.4]
	\foreach \x in {0,...,2} {
			\draw[->, blue] (\x, 0) -- ++(0, 2) \ifnum\x=1 node[right] {$\Bfield$}\fi;
			\draw[->, blue] ({\x+0.5}, -0.5) -- ++(0, 2);
			\draw[arrowpos={0.2}{2pt}{5pt}, red] (1.25, 0.5) coordinate (O) [partial ellipse=360:0:1.2 and 0.2] node[pos=0.7, circle, inner
					sep=1pt, ball
					color=red,
					text=white] (q)
			{$+$} ;
		}
	\draw[->, green!50!black, thick] (q) -- ++(3:1) node[above] {$\vect{v}$};
	\draw[gray, ->] (q) -- (O) node[anchor=west] {$\vect{F}$};
\end{tikzpicture}
		\end{column}
	\end{columns}
	\begin{block}{}
		За теоремою про зміну кінетичної енергії $A = \Delta\left(\frac{mv^2}{2}\right) = 0$, кінетична енергія частинки не змінюється.
	\end{block}
	\begin{alertblock}{}\centering
		Магнітне поле не виконує роботи над частинкою!
	\end{alertblock}
\end{frame}
% ===========================================================================



\end{document}