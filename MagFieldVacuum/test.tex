
\documentclass{beamer}
\usetheme{Electromagnetism}
\usepackage{Electromagnetism}
\graphicspath{{pictures/}}
% -------------------------------------- Grid
%-------------------------------------------------------
\makeatletter
\def\grd@save@target#1{%
  \def\grd@target{#1}}
\def\grd@save@start#1{%
  \def\grd@start{#1}}
\tikzset{
  grid with coordinates/.style={
    to path={%
      \pgfextra{%
        \edef\grd@@target{(\tikztotarget)}%
        \tikz@scan@one@point\grd@save@target\grd@@target\relax
        \edef\grd@@start{(\tikztostart)}%
        \tikz@scan@one@point\grd@save@start\grd@@start\relax
        \draw[minor help lines, gray!50] (\tikztostart) grid (\tikztotarget);
        \draw[major help lines, gray!50] (\tikztostart) grid (\tikztotarget);
        \grd@start
        \pgfmathsetmacro{\grd@xa}{\the\pgf@x/1cm}
        \pgfmathsetmacro{\grd@ya}{\the\pgf@y/1cm}
        \grd@target
        \pgfmathsetmacro{\grd@xb}{\the\pgf@x/1cm}
        \pgfmathsetmacro{\grd@yb}{\the\pgf@y/1cm}
        \pgfmathsetmacro{\grd@xc}{\grd@xa + \pgfkeysvalueof{/tikz/grid with coordinates/major step}}
        \pgfmathsetmacro{\grd@yc}{\grd@ya + \pgfkeysvalueof{/tikz/grid with coordinates/major step}}
        \foreach \x in {\grd@xa,\grd@xc,...,\grd@xb}
        \node[anchor=north] at (\x,\grd@ya) {\pgfmathprintnumber{\x}};
        \foreach \y in {\grd@ya,\grd@yc,...,\grd@yb}
        \node[anchor=east] at (\grd@xa,\y) {\pgfmathprintnumber{\y}};
      }
    }
  },
  minor help lines/.style={
    help lines,
    step=\pgfkeysvalueof{/tikz/grid with coordinates/minor step}
  },
  major help lines/.style={
    help lines,
    line width= 0.5pt,
    step=\pgfkeysvalueof{/tikz/grid with coordinates/major step}
  },
  grid with coordinates/.cd,
  minor step/.initial=.2,
  major step/.initial=1,
  major line width/.initial=2pt,
}
\makeatother

\begin{document}





% ============================== Слайд ## ===================================
\begin{frame}{Момент сили, що діє на контур в магнітному полі}{}\footnotesize
\begin{columns}
		\begin{column}{0.75\linewidth}
			\begin{block}{}\justifying
				Якщо виток перебуває в однорідному магнітному полі, то виникає момент сил, який орієнтує
				його магнітний момент за напрямком поля. За означенням моменту сил:
				\begin{equation*}
					\vect{M} = \frac1c \oint\limits_L  \vect{r}\times(Id\vect{\ell}\times\Bfield).
				\end{equation*}
 			\end{block}
		\end{column}
		\begin{column}{0.25\linewidth}\centering
			\begin{tikzpicture}[>=latex]
				\foreach \x in {-2,...,2} {
						\draw[->, blue] ({0.6*\x}, -1.5) -- ++(0, 3);
					}

				\begin{scope}[rotate around={45:(0,0)}]
					\draw[->, thick, gray] (0, 1) -- ++(-45:-0.7) node[below=5pt, inner sep=1pt, fill=white] {$d\vect{F}$};
					\draw[->, thick, gray] (0, -1) -- ++(-45:0.7);
					\draw[arrowpos={0.5}{2pt}{7pt}, red!40, ultra thick] (0,0) [partial ellipse=0:360:0.3 and 1];
                    \draw[red, ultra thick] (0,0) [partial ellipse=110:70:0.3 and 1];
					\draw[->,  thick] (0, 0) --  node[anchor=south west] {$\vect{r}'$} ++(0, 1);
				\end{scope}
			\end{tikzpicture}
		\end{column}
	\end{columns}
\begin{block}{}\tiny
    \alert{Треба витягнути $\Bfield$ з-під інтегралу. Всі інтеграли типу $\oint\limits_L d(\ldots) = 0$, як інтеграли повних диференціалів.}
    \begin{equation*}
        \overset{\color{red}a}{\vect{r}}\times(\overset{\color{red}b}{d\vect{r}}\times\overset{\color{red}c}{\Bfield}) =
        \overset{\color{red}b}{d\vect{r}}\ (\overset{\color{red}a}{\vect{r}}\cdot\overset{\color{red}c}{\Bfield}) -
        \overset{\color{red}c}{\Bfield}\ (\overset{\color{red}a}{\vect{r}}\cdot\overset{\color{red}b}{d\vect{r}}), \
        \oint\limits_L  d\vect{r} (\vect{r}\cdot\Bfield)  - \cancelto{0}{\Bfield \oint\limits_L d\left(\frac{r^2}{2}\right)}
    \end{equation*}
\begin{equation*}
    d\vect{r} (\vect{r}\cdot\Bfield) = \frac12 \left[d\vect{r} (\vect{r}\cdot\Bfield) + \vect{r} (d\vect{r}\cdot\Bfield) \right] +
    \frac12 \left[d\vect{r} (\vect{r}\cdot\Bfield) - \vect{r} (d\vect{r}\cdot\Bfield) \right] = \frac12 \left[d\vect{r} (\Bfield\cdot\vect{r}) +
    \vect{r} (\Bfield\cdot d\vect{r}) \right]  - \frac12 \Bfield\times(\vect{r}\times d\vect{r}).
\end{equation*}
\begin{equation*}
    d\vect{r} (\Bfield\cdot\vect{r}) +
    \vect{r} (\Bfield\cdot d\vect{r})  = d(\vect{r}(\Bfield\cdot\vect{r})),\
     \oint\limits_L d(\vect{r}(\Bfield\cdot\vect{r})) = 0.
\end{equation*}
\end{block}
\begin{equation*}
    \vect{M} = \left(\frac{I}{c} \oint\limits_L (\vect{r}\times d\vect{r})\right)\times\Bfield = \vect{p}_m\times\Bfield
\end{equation*}
\end{frame}
% ===========================================================================



\end{document}