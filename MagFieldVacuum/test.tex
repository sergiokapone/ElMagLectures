
\documentclass{beamer}
\usetheme{Electromagnetism}
\usepackage{Electromagnetism}
\graphicspath{{pictures/}}
\usetikzlibrary{spy}
% -------------------------------------- Grid
%-------------------------------------------------------
\makeatletter
\def\grd@save@target#1{%
  \def\grd@target{#1}}
\def\grd@save@start#1{%
  \def\grd@start{#1}}
\tikzset{
  grid with coordinates/.style={
    to path={%
      \pgfextra{%
        \edef\grd@@target{(\tikztotarget)}%
        \tikz@scan@one@point\grd@save@target\grd@@target\relax
        \edef\grd@@start{(\tikztostart)}%
        \tikz@scan@one@point\grd@save@start\grd@@start\relax
        \draw[minor help lines, gray!50] (\tikztostart) grid (\tikztotarget);
        \draw[major help lines, gray!50] (\tikztostart) grid (\tikztotarget);
        \grd@start
        \pgfmathsetmacro{\grd@xa}{\the\pgf@x/1cm}
        \pgfmathsetmacro{\grd@ya}{\the\pgf@y/1cm}
        \grd@target
        \pgfmathsetmacro{\grd@xb}{\the\pgf@x/1cm}
        \pgfmathsetmacro{\grd@yb}{\the\pgf@y/1cm}
        \pgfmathsetmacro{\grd@xc}{\grd@xa + \pgfkeysvalueof{/tikz/grid with coordinates/major step}}
        \pgfmathsetmacro{\grd@yc}{\grd@ya + \pgfkeysvalueof{/tikz/grid with coordinates/major step}}
        \foreach \x in {\grd@xa,\grd@xc,...,\grd@xb}
        \node[anchor=north] at (\x,\grd@ya) {\pgfmathprintnumber{\x}};
        \foreach \y in {\grd@ya,\grd@yc,...,\grd@yb}
        \node[anchor=east] at (\grd@xa,\y) {\pgfmathprintnumber{\y}};
      }
    }
  },
  minor help lines/.style={
    help lines,
    step=\pgfkeysvalueof{/tikz/grid with coordinates/minor step}
  },
  major help lines/.style={
    help lines,
    line width= 0.5pt,
    step=\pgfkeysvalueof{/tikz/grid with coordinates/major step}
  },
  grid with coordinates/.cd,
  minor step/.initial=.2,
  major step/.initial=1,
  major line width/.initial=2pt,
}
\makeatother

\begin{document}





% ============================== Слайд ## ===================================
\begin{frame}{Елемент струму}{}
\begin{onlyenv}<1>
	\begin{columns}
		\begin{column}{0.3\linewidth}\centering
			\begin{tikzpicture}[>=latex,
					spy using outlines={circle, magnification=3, size=3.2cm, connect spies}
				]
				\draw[gray!60, line width=0.21cm] (0,0)  [partial ellipse=90:0:2];
				\fill[gray!50, draw = gray!70] (0, 2) circle (0.05 and 0.1);
				\fill[gray!50, rotate=-90, draw = gray!70] (0, 2) circle (0.05 and 0.1);

				\draw[red!40, line width=0.21cm] (0,0) [partial ellipse=55:45:2];
				\draw[{Latex[scale=0.5]}-{Latex[scale=0.5]}] (55:2.2)  arc(55:45:2.1) node[above=-1pt, sloped, pos=0.7, font=\tiny] {$d\ell$};
				\fill[red!40, draw = red!60, rotate=-45] (0, 2) coordinate (S1) circle (0.05 and 0.1);
				\fill[red!40, draw = red!60, rotate=-35] (0, 2)  circle (0.05 and 0.1);

				\node[font=\tiny, right, inner sep=0] (S) at (42:2.2) {$S$};

				\draw[{Latex[scale=0.3]}-, ultra thin] (S1) to[out=0] (S.west);

				\draw[-{Latex[scale=0.5]}, teal] (55:2)  node[below=1pt, text=teal, font=\tiny] {$\vect{j}$} -- ++({55-90}:0.2);
				\spy [gray] on (50:2.1)
				in node [] at (1.5,4.5);
			\end{tikzpicture}
		\end{column}
		\begin{column}{0.7\linewidth}
			\begin{block}{}\justifying
				Якщо в задачі не цікавляться
				внутрішньою будовою провідника, та розподілом струму в його товщі, то можна ввести \alert{лінійний елемент струму}.
			\end{block}
			\begin{block}{}\justifying\footnotesize
				Нехай струм тече провідником із площею поперечного перерізу $S$. Уведемо вектор ділянки провідника завдовжки $d\vect{\ell}$
				за формулою $d\vect{\ell} = \vect{n}\ell$, де $\vect{n}$ --- одиничний вектор уздовж осі провідника. Тоді $\vect{j} = j\vect{n}$,
				а $I = j S$ і вираз для об'ємного елемента струму можна переписати у вигляді:
				\begin{equation*}
					\vect{j}dV = j\vect{n} S d\ell  = I d\vect{\ell}.
				\end{equation*}
			\end{block}
		\end{column}
	\end{columns}
	\begin{block}{}
		Для лінійного елемента струму сила Ампера:
		\begin{equation*}
			d\vect{F} = \frac1c \left[ I d\vect{\ell} \times \Bfield\right].
		\end{equation*}
	\end{block}
\end{onlyenv}
\begin{onlyenv}<2>
\begin{tblr}{
colspec={Q[c,h]X[l,m]}
}
\tikz[>=latex, baseline]{
	\draw[gray!60, line width=0.21cm] (0,0)  [partial ellipse=90:0:2];
	\fill[gray!50, draw = gray!70] (0, 2) circle (0.05 and 0.1);
	\fill[gray!50, rotate=-90, draw = gray!70] (0, 2) circle (0.05 and 0.1);

	\draw[red!40, line width=0.21cm] (0,0) [partial ellipse=55:45:2];
	\draw[{Latex[scale=0.5]}-{Latex[scale=0.5]}] (55:2.2)  arc(55:45:2.1) node[above=-1pt, sloped, pos=0.7, font=\tiny] {$d\ell$};
	\fill[red!40, draw = red!60, rotate=-45] (0, 2) coordinate (S1) circle (0.05 and 0.1);
	\fill[red!40, draw = red!60, rotate=-35] (0, 2)  circle (0.05 and 0.1);
	\node[font=\tiny, right, inner sep=0] (S) at (42:2.2) {$S$};
	\draw[{Latex[scale=0.3]}-, ultra thin] (S1) to[out=0] (S.west);
	\draw[-{Latex[scale=0.5]}, teal] (55:2)  node[below=1pt, text=teal, font=\tiny] {$\vect{j}$} -- ++({55-90}:0.2);
}
&
Елемент об'ємного струму $\vect{j}dV$
\\
\tikz[>=latex, baseline]{
	\draw[gray!60, line width=0.05cm, arrowpos={0.2}{2pt}{6pt}] (0,0)  [partial ellipse=90:0:2] node[pos=0.2, anchor=north, text=black] {$I$};
	\draw[red!40, line width=0.05cm] (0,0) [partial ellipse=55:45:2];
	\draw[-{Latex[scale=0.5]}] (55:2)  arc(55:45:2) node[above, sloped, pos=0.7, font=\tiny] {$d\vect{\ell}$};
}
&
Елемент лінійного струму $Id\vect{\ell}$
\\
\tikz[>=latex, baseline,
                    pencildraw/.style={ %
							decorate,
							decoration={random steps,segment length=2pt,amplitude=1pt}
						},]{
        \draw[fill=gray!60] (0,0) -- ++(3, 0) decorate [pencildraw]
          {to ++(1, 1)}
        -- ++(-3, 0)
         decorate [pencildraw]  {to cycle};
        \foreach \s in {0.2,0.4,...,0.8}  {
            \draw[->, red] ({0.8+\s}, \s) -- ++(1.2, 0);
        }
        \draw (1.5, 0) -- node[right, fill=gray!60, circle, inner sep=0] {$l$} ++(1,1) node (S) [above=5pt] {$dS$};
        \draw[fill=red!50, opacity=0.5] (1.0,0) -- ++(1, 0) -- ++(1, 1) -- ++(-1, 0) -- cycle;
        \draw[->] (S.east) to[out=0, in=90] (2.7, 0.9);

}
&
{
Поверхнева густина струму $i = \frac{I}{l}$.\\
Елемент струму $I\ell = il\ell = i S$, де $\ell$ та $l$ --- сторони виділеного елемента, площа якого $S = l\cdot \ell$.
Елемент поверхневого струму $\vect{i}dS$
}
\end{tblr}
\end{onlyenv}
\end{frame}
% ===========================================================================



\end{document}