
\documentclass[onlytextwidth]{beamer}
\usetheme{Electromagnetism}
\usepackage{Electromagnetism}
\graphicspath{{pictures/}}
% -------------------------------------- Grid
%-------------------------------------------------------
\makeatletter
\def\grd@save@target#1{%
  \def\grd@target{#1}}
\def\grd@save@start#1{%
  \def\grd@start{#1}}
\tikzset{
  grid with coordinates/.style={
    to path={%
      \pgfextra{%
        \edef\grd@@target{(\tikztotarget)}%
        \tikz@scan@one@point\grd@save@target\grd@@target\relax
        \edef\grd@@start{(\tikztostart)}%
        \tikz@scan@one@point\grd@save@start\grd@@start\relax
        \draw[minor help lines] (\tikztostart) grid (\tikztotarget);
        \draw[major help lines] (\tikztostart) grid (\tikztotarget);
        \grd@start
        \pgfmathsetmacro{\grd@xa}{\the\pgf@x/1cm}
        \pgfmathsetmacro{\grd@ya}{\the\pgf@y/1cm}
        \grd@target
        \pgfmathsetmacro{\grd@xb}{\the\pgf@x/1cm}
        \pgfmathsetmacro{\grd@yb}{\the\pgf@y/1cm}
        \pgfmathsetmacro{\grd@xc}{\grd@xa + \pgfkeysvalueof{/tikz/grid with coordinates/major step}}
        \pgfmathsetmacro{\grd@yc}{\grd@ya + \pgfkeysvalueof{/tikz/grid with coordinates/major step}}
        \foreach \x in {\grd@xa,\grd@xc,...,\grd@xb}
        \node[anchor=north] at (\x,\grd@ya) {\pgfmathprintnumber{\x}};
        \foreach \y in {\grd@ya,\grd@yc,...,\grd@yb}
        \node[anchor=east] at (\grd@xa,\y) {\pgfmathprintnumber{\y}};
      }
    }
  },
  minor help lines/.style={
    help lines,
    step=\pgfkeysvalueof{/tikz/grid with coordinates/minor step}
  },
  major help lines/.style={
    help lines,
    line width= 0.5pt,
    step=\pgfkeysvalueof{/tikz/grid with coordinates/major step}
  },
  grid with coordinates/.cd,
  minor step/.initial=.2,
  major step/.initial=1,
  major line width/.initial=2pt,
}
\makeatother

\begin{document}
% Butikov E.I., Kondrat'ev A.S. Fizika.. Ucheb. posobie.. Kn. 2. E'lektrodinamika. Optika (FML, 2004)(ISBN 5922101102)(ru)(K)(T)(335s)_PSch_

% ============================== Слайд ## ===================================
\begin{frame}[fragile]{Випромінювання заряду}{}
	\begin{block}{}\justifying
		Розглянемо електричне поле, точкового заряду \( q \).
		Якщо заряд перебуває в стані спокою, його електростатичне поле описується
		радіальними силовими лініями, що виходять із заряду.
	\end{block}
	\begin{columns}
		\begin{column}{0.4\linewidth}\centering
			\begin{block}{}\justifying
				Нехай у момент часу \( t = 0 \) заряд під дією зовнішньої сили
				починає рухатися з прискоренням \( a \), а через деякий час \( \tau \)
				дія сили припиняється, після чого заряд рухається рівномірно зі
				швидкістю \( v = a\tau \). Графік швидкості руху заряду наведено на рис.
			\end{block}
			\begin{tikzpicture}[>=latex]
				\draw[->] (0,0) -- ++(0, 1) node[left] {$v_q$};
				\draw[->] (0,0) -- ++(2.5, 0) node[below] {$t$};
				\draw[red, thick] (0,0) -- ++(0.25, 0.75) coordinate (tau)  -- ++(1.75, 0) coordinate (T) -- ++(0.25,0);
				\draw[dashed] (tau) -- ++(0,-0.75) node[below] {$\tau$};
				\draw[dashed] (T) -- ++(0,-0.75) node[below] {$T$};
				\node[above] at (T) {$v_q \ll c$};
				\node[below] at ($(0.75,0)!0.3!(2, 0)$) {$\tau \ll T$};
			\end{tikzpicture}
		\end{column}
		\begin{column}{0.6\linewidth}\centering
			\begin{tikzpicture}[
					>=latex,
					chp/.style = {
							circle,
							inner sep=1pt,
							ball color=red,
							font=\tiny\mathstrut\bfseries,
							text=white,
							inner sep=0pt
						},
					mydash/.style = {
							dash pattern={on 1pt off 0.5pt}
						},
					myarrow/.style = {
							->,
							red
						},
					mytext/.style = {
							align=center,
							text width=#1,
							font=\scriptsize\itshape,
							inner sep=0
						}
				]

				% Основні параметри
				\def\r{1.8} % Внутрішній радіус
				\def\R{2}   % Зовнішній радіус
				\def\d{0.5} % Вертикальне зміщення заряду
				\def\arrowLength{1}   % Довжина стрілок
				\def\angles{0,22.5,...,340} % Кути для ліній
				\uncover<1>{
					\node[chp] (q) at (0,0) {$+$};
					% Стрілки та лінії
					\foreach[count=\i] \angle in \angles {
						\draw[myarrow] (q) -- ++(\angle:{\R + \arrowLength});
					}
				}
%				\uncover<2>{
%					\node[chp] (q') at (0,\d) {$+$};
%					\foreach[count=\i] \angle in \angles {
%						\draw[red, ->] (q') -- ++(\angle:{\R + \arrowLength});
%					}}
				\uncover<2>{
				% Основні вузли
				\node[chp, opacity=0.5] (q) at (0,0) {$+$};
				\node[chp] (q') at (0,\d) {$+$};

				% Кола
				\draw[gray, mydash, name path=innerCircle] (q) circle(\r);
				\draw[gray, mydash] (q) circle(\R);

				% Лінії всередині внутрішнього кола
				\begin{scope}
					\clip (q) circle(\r);
					\foreach[count=\i] \angle in \angles {
						\draw[red, opacity=0.5, name path global=line\i] (q') -- ++(\angle:{\R + \d});
					}
				\end{scope}

				% Стрілки та лінії
				\foreach[count=\i] \angle in \angles {
					\draw[myarrow] (q) ++(\angle:\R) coordinate (endOfLine\i) -- ++(\angle:\arrowLength);
					\draw[red, name intersections={of=innerCircle and line\i}] (intersection-1) -- (endOfLine\i);
				}

				% Координати для спостерігачів
				\coordinate (innerViewer) at (-40:{\r-0.5});
				\coordinate (outerViewer) at (220:{\R+0.5});
				\coordinate (accRegion) at (265:{(\R+\r)/2});

				% Текст та стрілки для спостерігачів
				\node[mytext=1.2cm] (textOuterViewer) at (220:{\R+1.5}) {Зовнішній\\ спостерігач};
				\draw[->, gray!80] (textOuterViewer) to[bend left] (outerViewer);

				\node[mytext=1.2cm] (textInnerViewer) at (-45:{\R+1.5}) {Внутрішній\\ спостерігач};
				\draw[->,  gray!80] (textInnerViewer)  to[bend left] (innerViewer);

				\node[mytext=3cm] (textAccRegionViewer) at (-90:{\R+1.6}) {Область випромінювання \\ під час прискорення};
				\draw[->,  gray!80] (textAccRegionViewer)  to[bend right] (accRegion);
				}

			\end{tikzpicture}
		\end{column}
	\end{columns}
\end{frame}
% ===========================================================================

% ============================== Слайд ## ===================================
\begin{frame}{Математичні викладки}{}

	\begin{columns}
		\begin{column}{0.5\linewidth}\centering
			\begin{tikzpicture}[
					>=latex,
					chp/.style = {
							circle,
							inner sep=1pt,
							ball color=red,
							font=\tiny\mathstrut\bfseries,
							text=white,
							inner sep=0pt
						},
					mydash/.style = {
							dash pattern={on 1pt off 0.5pt}
						},
					myarrow/.style = {
							->,
							red
						},
					mybrace/.style = {
							decorate,
							decoration={brace, amplitude=0.5ex, raise=0.1ex}
						}
				]

				% Определение параметров
				\def\r{2} % Внутренний радиус
				\def\R{2.4} % Внешний радиус
				\def\d{0.75} % Вертикальное смещение
				\def\angle{35} % Угол
				\def\arrowLength{1} % Длина стрелки
                \pgfmathsetmacro\ctang{(\d*cos(\angle)) / (\R-sqrt(\r^2-\d^2*cos(\angle)))}
                \pgfmathsetmacro\E{sqrt(1 + \ctang^2}

				% Основные узлы
				\node[chp, opacity=0.5] (q) at (0,0) {$+$};
				\node[chp] (q') at (0,\d) {$+$};

				% Окружности
				\draw[gray, mydash, name path=innerCircle] (q) circle(\r);
				\draw[gray, mydash] (q) circle(\R);

				% Обрезка и рисование линии
				\begin{scope}
					\clip (q) circle(\r);
					\draw[red, opacity=0.5, name path global=redLine] (q') -- ++(\angle:{\R + \d});
				\end{scope}

				% Стрелки и линии
				\draw[myarrow] (q) ++(\angle:\R) coordinate (endOfRedLine) -- ++(\angle:\arrowLength);
				\draw[red, name intersections={of=innerCircle and redLine}] (intersection-1) coordinate (startOfRedLine) -- (endOfRedLine);

				% Пунктирные линии и углы
				\draw[dashed] (q) -- ++(\angle:\R);
				\draw[mybrace] (q.west) -- node[left] {$v_qT$} (q'.west);
				\draw[mydash] (q') -- coordinate (projectionPoint) ++({\angle-90}:{\d*cos(\angle)});
				\draw (q') -- ++(0, 2);
				\draw (q') ++(0, 0.5) arc(90:\angle:0.5) node[pos=0.5, anchor=south west, inner sep=0] {$\theta$};

				% Стрелки и подписи
				\draw[->] (295:{\r - 0.25}) node[above] (cTauLabel) {$c\tau$} -- (295:\r);
				\draw[->] (295:{\R + 0.25})                                   -- (295:\R);
				\draw (q) -- ++(225:\R) node[above, pos=0.5, sloped] {$r = cT$};
				\draw (q) -- (265:\r) node[above, pos=0.5, sloped] {$c(T-\tau)$};
                \node[text=black] (vqTSinThetaLabel) at (1, -0.5) {$v_qT\sin\theta$};
				\draw[thick] (startOfRedLine) -- coordinate (midPoint)  ++({\angle-90}:{\d*sin(90-\angle)});
				\draw[->, gray!70] (vqTSinThetaLabel.north) to[in=0] (projectionPoint);
				\draw[->, gray!70] (vqTSinThetaLabel.north) to[in=225] (midPoint);
				\draw[thick] (\angle:\r) -- coordinate (endPoint) (endOfRedLine);
				\draw[->, gray!70] (cTauLabel) to[in=-85] (endPoint);

				\draw[->, thick, red] (endOfRedLine) -- ++(\angle:1) coordinate (Epl) node[above]  {$\Efield_{\parallel}$};
				\draw[->, thick, red] (endOfRedLine) -- ++({\angle-90}:{\ctang}) coordinate (Epr) node[below] {$\Efield_{\perp}$};
				\draw[->, thick, red] (endOfRedLine) -- ++({\angle-atan(\ctang)}:{\E}) coordinate (E)
				node[right] {$\Efield$};
				\draw[mydash] (Epl) -- (E) -- (Epr);


			\end{tikzpicture}
		\end{column}
		\begin{column}{0.5\linewidth}
			\begin{equation*}
				\frac{E_{\perp}}{E_{\parallel}} = \frac{v_q T\sin\theta}{c\tau}
			\end{equation*}
			\begin{equation*}
				E_{\parallel} = \frac{q}{(cT)^2} \stackrel{r = cT}{=} \frac{q}{(cT) r}
			\end{equation*}
			\begin{equation*}
				a= \frac{v_q}{\tau}
			\end{equation*}
			\begin{equation*}
				E_{\perp}(t) = \frac{q}{(cT) r}  \frac{v_q T\sin\theta}{c\tau}
			\end{equation*}
			\begin{equation*}
				E_{\perp}(t) = \frac1{c^2} \frac{q}{r} a\left( t - \frac{r}{c}\right) \sin\theta
			\end{equation*}
		\end{column}
	\end{columns}
	\begin{block}{}\justifying
		Напруженість електричного поля хвилі \( E_\perp \) спадає як \( {1}/{r} \),
		на відміну від електростатичного поля \( E_\parallel \), яке спадає як \( {1}/{r^2} \).
		Це пояснюється законом збереження енергії: енергія хвилі розподіляється по поверхні сфери (\( \propto r^2 \)),
		а густина енергії \( \propto E^2 \). Таким чином, \( E_\perp \propto {1}/{r} \).
		Крім того, \( E_\perp \) у момент часу \( t \) залежить від прискорення заряду \( a \)
		в момент \( t - {r}/{c} \), оскільки хвиля досягає точки через час \( {r}/{c} \).
	\end{block}
\end{frame}
% ===========================================================================

\end{document}