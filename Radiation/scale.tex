
% !TeX program = lualatex
% !TeX encoding = utf8
% !TeX spellcheck = uk_UA

\documentclass[14pt]{extarticle}

\usepackage{fontspec}
\setsansfont{CMU Sans Serif}%{Arial}
\setmainfont{CMU Serif}%{Times New Roman}
\setmonofont{CMU Typewriter Text}%{Consolas}
\defaultfontfeatures{Ligatures={TeX}}
\usepackage[math-style=TeX]{unicode-math}
\usepackage[english, russian, ukrainian]{babel}


\usepackage[%
	a4paper,%
	footskip=1cm,%
	headsep=0.3cm,%
	top=2cm, %поле сверху
	bottom=2cm, %поле снизу
	left=2cm, %поле ліворуч
	right=2cm, %поле праворуч
    ]{geometry}

\renewcommand{\baselinestretch}{1}


\setlength{\parskip}{0.5ex}%
\setlength{\parindent}{2.5em}%

\usepackage{amsmath}
\usepackage{graphicx}
\usepackage{floatflt}

\usepackage[%colorlinks=true,
	%urlcolor = blue, %Colour for external hyperlinks
	%linkcolor  = malina, %Colour of internal links
	%citecolor  = green, %Colour of citations
	bookmarks = true,
	bookmarksnumbered=true,
	unicode,
	linktoc = all,
	hypertexnames=false,
	pdftoolbar=false,
	pdfpagelayout=TwoPageRight,
	pdfauthor={Ponomarenko S.M. aka sergiokapone},
	pdfdisplaydoctitle=true,
	pdfencoding=auto
	]%
	{hyperref}
		\makeatletter
	\AtBeginDocument{
	\hypersetup{
		pdfinfo={
		Title={\@title},
		}
	}
	}
	\makeatother

\title{}
\author{}

\usepackage{tabularray}

\begin{document}
\begin{table}[h!]
\centering
\caption{Характеристики электромагнитных волн}
\begin{tblr}{
  colspec={Q[l,m, 3cm]Q[c,m]Q[c,m]Q[l,m, 4cm]},
  hlines,
  vlines,
  rowsep=2pt,
  colsep=4pt
}
    Диапазон                            & Длина волны & Частота         & Источник волн                     \\
    Радиоволны                          & 100--10 км           & 3--30 кГц          & Радиостанции, атмосферные разряды          \\
    Сверхдлинные волны (СДВ)            & 10--1 км             & 30--300 кГц        & Подводные передатчики, антенны             \\
    Длинные волны (ДВ)                  & 1 км--100 м          & 0,3--3 МГц         & Радиовещание, природные процессы           \\
    Короткие волны (КВ)                 & 100--10 м            & 3--30 МГц          & Радиосвязь, передатчики                    \\
    Ультракороткие волны (УКВ)          & 10 м--1 мм           & 30 МГц--300 ГГц    & Телевещание, мобильная связь               \\
    Инфракрасное (ИК) излучение         & 1 мм--0,76 мкм       & 300 ГГц--0,39 ПГц  & Тепловое излучение, нагревательные приборы \\
    Оптическое излучение (видимый свет) & 0,76 мкм--0,4 мкм    & 0,39 ПГц--0,75 ПГц & Солнце, лампы накаливания
\end{tblr}
\end{table}

\end{document}


