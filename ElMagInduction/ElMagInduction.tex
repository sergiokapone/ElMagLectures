% !TeX program = lualatex
% !TeX encoding = utf8
% !TeX spellcheck = uk_UA
% !BIB program = bibler

\documentclass[onlytextwidth]{beamer}
\usetheme{Electromagnetism}
\usepackage{Electromagnetism}
\usepackage{circuitikz}


%============================================================================
\title[Лекції електрики та магнетизму]{\huge\bfseries Явище електромагнітної індукції}
\subtitle{Лекції з електрики та магнетизму}
\author{Пономаренко С. М.}
\date{}
%============================================================================
\graphicspath{{pictures/}}
\begin{document}
\begin{frame}[plain]
	\maketitle
\end{frame}

% ============================== Слайд ## ===================================
\begin{frame}{Зміст}{}
	\tableofcontents
\end{frame}
% ===========================================================================

% ============================== Слайд ## ===================================
\begin{frame}{Явище електромагнітної індукції}{}
\begin{block}{Явище електромагнітної індукції (Фарадей)}\justifying
У 1831 р. Фарадеєм було зроблено одне з найбільш фундаментальних відкриттів в електродинаміці --- \alert{явище електромагнітної індукції}. Воно
полягає в тому, що в замкненому провідному контурі при зміні магнітного потоку, охопленого цим контуром, виникає електричний струм --- його назвали
індукційним.
\end{block}
\end{frame}
% ===========================================================================



% ============================== Слайд ## ===================================
\begin{frame}{Досліди Фарадея}{}

\end{frame}
% ===========================================================================

\end{document}
