
\documentclass{beamer}
\usetheme{Electromagnetism}
\usepackage{Electromagnetism}
\graphicspath{{pictures/}}
% -------------------------------------- Grid
%-------------------------------------------------------
\makeatletter
\def\grd@save@target#1{%
  \def\grd@target{#1}}
\def\grd@save@start#1{%
  \def\grd@start{#1}}
\tikzset{
  grid with coordinates/.style={
    to path={%
      \pgfextra{%
        \edef\grd@@target{(\tikztotarget)}%
        \tikz@scan@one@point\grd@save@target\grd@@target\relax
        \edef\grd@@start{(\tikztostart)}%
        \tikz@scan@one@point\grd@save@start\grd@@start\relax
        \draw[minor help lines] (\tikztostart) grid (\tikztotarget);
        \draw[major help lines] (\tikztostart) grid (\tikztotarget);
        \grd@start
        \pgfmathsetmacro{\grd@xa}{\the\pgf@x/1cm}
        \pgfmathsetmacro{\grd@ya}{\the\pgf@y/1cm}
        \grd@target
        \pgfmathsetmacro{\grd@xb}{\the\pgf@x/1cm}
        \pgfmathsetmacro{\grd@yb}{\the\pgf@y/1cm}
        \pgfmathsetmacro{\grd@xc}{\grd@xa + \pgfkeysvalueof{/tikz/grid with coordinates/major step}}
        \pgfmathsetmacro{\grd@yc}{\grd@ya + \pgfkeysvalueof{/tikz/grid with coordinates/major step}}
        \foreach \x in {\grd@xa,\grd@xc,...,\grd@xb}
        \node[anchor=north] at (\x,\grd@ya) {\pgfmathprintnumber{\x}};
        \foreach \y in {\grd@ya,\grd@yc,...,\grd@yb}
        \node[anchor=east] at (\grd@xa,\y) {\pgfmathprintnumber{\y}};
      }
    }
  },
  minor help lines/.style={
    help lines,
    step=\pgfkeysvalueof{/tikz/grid with coordinates/minor step}
  },
  major help lines/.style={
    help lines,
    line width= 0.5pt,
    step=\pgfkeysvalueof{/tikz/grid with coordinates/major step}
  },
  grid with coordinates/.cd,
  minor step/.initial=.2,
  major step/.initial=1,
  major line width/.initial=2pt,
}
\makeatother
\usepackage{cancel}
\begin{document}



% ============================== Слайд ## ===================================
\begin{frame}{Граничні умови}{}\small
	\begin{tikzpicture}[>=latex, scale=0.6, transform shape]
		\begin{scope}
			\draw[red, line width=2pt] (0,0) circle[x radius=1cm, y radius=0.3cm];
			\foreach[count=\i from -2] \x in {-0.8,-0.4,...,1} {
					\draw[->, blue, thick] (\x, -2) -- ++(0,4) \ifnum\i=0 node[above] {$\Bfield$}\fi;
				}
			\draw[red, line width=2pt, arrowpos={0.5}{3pt}{7pt}] (0,0) ++(180:1) arc[start angle=180, end angle=360, x radius=1cm, y radius=0.3cm];
		\end{scope}
		\begin{scope}[yshift=-4.3cm]
			\draw[red, line width=2pt] (1,0) circle[x radius=1cm, y radius=0.3cm];
			\foreach[count=\i from -2] \x in {-0.8,-0.4,...,1} {
					\begin{scope}[blue, thick, xshift=\x cm]
						\draw [in=135, out=-90, <-] (0, 2) to (0.25, 1);
						\draw [in=45, out=-45]  (0.25, 1) to (0.75, -0.5);
						\draw [in=90, out=-135] (0.75, -0.5) to (0, -2);
					\end{scope}
				}
			\draw[red, line width=2pt, arrowpos={0.5}{3pt}{7pt}] (1,0) ++(180:1) arc[start angle=180, end angle=360, x radius=1cm, y radius=0.3cm];
			\draw[->, thick] (-0.7, 0) -- ++(0.5, 0);
		\end{scope}
	\end{tikzpicture}
\end{frame}
% ===========================================================================

\end{document}